
\documentclass{lecture}
\DeclareMathOperator*{\argmax}{arg\,max}
\DeclareMathOperator*{\argmin}{arg\,min}

\usepackage{algorithmicx}

\usepackage{tikz-dependency}
\usepackage{placeins}
\usepackage{xcolor}

\title{Non-monotonic state transitions for dependency parsing}
\author{Matthew Honnibal, Yoav Goldberg and Mark Johnson}


\begin{document}

\titleslide

\begin{plain}{Before (above) and After (below) Non-monotonic Left-Arc}


\begin{figure}
    \centering
    \begin{dependency}[theme=simple]
        \tikzstyle{t}=[text=green,ultra thick,font=\bfseries\itshape]
        \tikzstyle{f}=[text=red,ultra thick,font=\bfseries\itshape]
        \tikzstyle{m}=[font=\bfseries\itshape]
        \tikzstyle{n}=[font=\itshape]

        \begin{deptext}[column sep=.075cm, row sep=.1ex]
            I \&           \& saw \&          \& Jack       \& \& and     \&           \& Jill   \&  \& fall \& \textsc{r} \\
|[m]|S \& |[m]|L \& |[m]|S   \& |[f]|R \& |[m]|R \& |[m]|D \& |[m]|R \& |[m]|D \& |[t]|NM-L \\
    1 \&     2       \& 3  \&   4      \& 5          \& 6 \& 7     \& 8 \& 9 \& 10 \& 11 \& 12 \\
    I \&           \& saw \&          \& Jack       \& \& and     \&           \& Jill   \&      \& fall \& \textsc{r} \\
        \end{deptext}
    \depedge{3}{1}{2}
    \depedge[red, ultra thick]{3}{5}{4}
    \depedge{5}{7}{5}
    \depedge{5}{9}{7}
    
    \deproot[edge height=0.7cm, ultra thick]{11}{}
    \wordgroup{1}{3}{3}{}
    \wordgroup{1}{5}{5}{}
    
    \depedge[edge below]{3}{1}{2}
    \depedge[edge below]{5}{7}{5}
    \depedge[edge below]{5}{9}{7}
    \depedge[edge below, green, ultra thick]{11}{5}{9}
\end{dependency}
\end{figure}

We allow the Left-Arc to `clobber' the existing head of \emph{Jack},  \emph{saw}.
It sets \emph{fall} as the head of \emph{Jack} instead. 

\end{plain}



\begin{points}{Abstract}
    \p Previous incremental (e.g. left-to-right) parsers have used
    \emph{monotonic} transition systems
    \p Monotonicity means the parser \emph{commits} to each action
    \p Previous actions restrict the choice of future actions\\
    \vspace*{.2in}
    \par
    \p We introduce a limited amount of `non-monotonic' behaviour
    \p i.e. an action might be inconsistent with previous decisions
    \p Accuracy improves for a greedy parser on multiple languages
    \p 91.1\% English UAS, at over 1,000 sentences per second
\end{points}


\begin{plain}{The Arc-Eager Transition System}
\begin{figure}
    \centering
    \begin{dependency}[theme=simple]
    \tikzstyle{t}=[text=green,ultra thick,font=\bfseries\itshape]
    \tikzstyle{f}=[text=red,ultra thick,font=\bfseries\itshape]
    \tikzstyle{m}=[font=\bfseries\itshape]
    \tikzstyle{n}=[font=\itshape]
    \begin{deptext}[column sep=.075cm, row sep=.1ex]
        I \&           \& saw \&          \& Jack       \& \& fall   \& \& \textsc{r} \\
         \& \\
    \end{deptext}
    \deproot[edge height=0.7cm, ultra thick]{1}{}
\end{dependency}
\end{figure}
    \begin{itemize}
        \item \textbf{S}hift: \emph{Push} the first word of the buffer onto the stack
    \item \textbf{R}ight-Arc: Set top-of-stack as Head of start-of-buffer + \emph{Push}.
    \item Re\textbf{D}uce: \emph{Pop} the top word off the stack
    \item \textbf{L}eft-Arc: Set start-of-buffer as Head of top-of-stack + \emph{Pop}.
    \end{itemize}

\end{plain}



\begin{plain}{Parse example. Words on stack circled, arrow shows buffer}
\begin{figure}
    \centering
    \begin{dependency}[theme=simple]
    \tikzstyle{t}=[text=green,ultra thick,font=\bfseries\itshape]
    \tikzstyle{f}=[text=red,ultra thick,font=\bfseries\itshape]
    \tikzstyle{m}=[font=\bfseries\itshape]
    \tikzstyle{n}=[font=\itshape]
    \begin{deptext}[column sep=.075cm, row sep=.1ex]
        I \&           \& saw \&          \& Jack       \& \& fall   \& \& \textsc{r} \\
        S \& \\
    \end{deptext}
    \wordgroup{1}{1}{1}{}
    \deproot[edge height=0.7cm, ultra thick]{3}{}
\end{dependency}
\end{figure}
    \begin{itemize}
        \item \textbf{S}hift: \emph{Push} the first word of the buffer onto the stack
    \item \textbf{R}ight-Arc: Set top-of-stack as Head of start-of-buffer + \emph{Push}.
    \item Re\textbf{D}uce: \emph{Pop} the top word off the stack
    \item \textbf{L}eft-Arc: Set start-of-buffer as Head of top-of-stack + \emph{Pop}.
    \end{itemize}

\end{plain}


\begin{plain}{Parse example. Words on stack circled, arrow shows buffer}

\begin{figure}
    \centering
    \begin{dependency}[theme=simple]
    \tikzstyle{t}=[text=green,ultra thick,font=\bfseries\itshape]
    \tikzstyle{f}=[text=red,ultra thick,font=\bfseries\itshape]
    \tikzstyle{m}=[font=\bfseries\itshape]
    \tikzstyle{n}=[font=\itshape]
    \begin{deptext}[column sep=.075cm, row sep=.1ex]
        I \&           \& saw \&          \& Jack       \& \& fall   \& \& \textsc{r} \\
      S   \& L \\ 
    \end{deptext}
    \depedge{3}{1}{}
    \deproot[edge height=0.7cm, ultra thick]{3}{}
\end{dependency}
\end{figure}
    \begin{itemize}
        \item \textbf{S}hift: \emph{Push} the first word of the buffer onto the stack
    \item \textbf{R}ight-Arc: Set top-of-stack as Head of start-of-buffer + \emph{Push}.
    \item Re\textbf{D}uce: \emph{Pop} the top word off the stack
    \item \textbf{L}eft-Arc: Set start-of-buffer as Head of top-of-stack + \emph{Pop}.
    \end{itemize}

\end{plain}


\begin{plain}{Parse example. Words on stack circled, arrow shows buffer}
\begin{figure}
    \centering
    \begin{dependency}[theme=simple]
    \tikzstyle{t}=[text=green,ultra thick,font=\bfseries\itshape]
    \tikzstyle{f}=[text=red,ultra thick,font=\bfseries\itshape]
    \tikzstyle{m}=[font=\bfseries\itshape]
    \tikzstyle{n}=[font=\itshape]
    \begin{deptext}[column sep=.075cm, row sep=.1ex]
        I \&           \& saw \&          \& Jack       \& \& fall   \& \& \textsc{r} \\
      S   \& L \& S \\ 
    \end{deptext}
    \depedge{3}{1}{}
    \depedge{3}{1}{}
    \deproot[edge height=0.7cm, ultra thick]{5}{}
    \wordgroup{1}{3}{3}{}
\end{dependency}
\end{figure}
    \begin{itemize}
        \item \textbf{S}hift: \emph{Push} the first word of the buffer onto the stack
    \item \textbf{R}ight-Arc: Set top-of-stack as Head of start-of-buffer + \emph{Push}.
    \item Re\textbf{D}uce: \emph{Pop} the top word off the stack
    \item \textbf{L}eft-Arc: Set start-of-buffer as Head of top-of-stack + \emph{Pop}.
    \end{itemize}
\end{plain}


\begin{plain}{Parse example. Words on stack circled, arrow shows buffer}
\begin{figure}
    \centering
    \begin{dependency}[theme=simple]
    \tikzstyle{t}=[text=green,ultra thick,font=\bfseries\itshape]
    \tikzstyle{f}=[text=red,ultra thick,font=\bfseries\itshape]
    \tikzstyle{m}=[font=\bfseries\itshape]
    \tikzstyle{n}=[font=\itshape]
    \begin{deptext}[column sep=.075cm, row sep=.1ex]
        I \&           \& saw \&          \& Jack       \& \& fall   \& \& \textsc{r} \\
        S   \& L \& S \& |[f]|R \\ 
    \end{deptext}
    \depedge{3}{1}{}
    \depedge[red]{3}{5}{}
    \deproot[edge height=0.7cm, ultra thick]{7}{}
    \wordgroup{1}{3}{3}{}
    \wordgroup{1}{5}{5}{}
\end{dependency}
\end{figure}
    \begin{itemize}
        \item \textbf{S}hift: \emph{Push} the first word of the buffer onto the stack
    \item \textbf{R}ight-Arc: Set top-of-stack as Head of start-of-buffer + \emph{Push}.
    \item Re\textbf{D}uce: \emph{Pop} the top word off the stack
    \item \textbf{L}eft-Arc: Set start-of-buffer as Head of top-of-stack + \emph{Pop}.
    \end{itemize}
\end{plain}


\begin{plain}{Arc-Eager Example. Words on stack circled, arrow shows buffer}
\begin{figure}
    \centering
    \begin{dependency}[theme=simple]
    \tikzstyle{t}=[text=green,thick,font=\bfseries\itshape]
    \tikzstyle{f}=[text=red,ultra thick,font=\bfseries\itshape]
    \tikzstyle{m}=[font=\bfseries\itshape]
    \tikzstyle{n}=[font=\itshape]
    \begin{deptext}[column sep=.075cm, row sep=.1ex]
        I \&           \& saw \&          \& Jack       \& \& fall   \& \& \textsc{r} \\
        S   \& L \& S \& |[f]|R \& *L \&  \\ 
    \end{deptext}
    \depedge{3}{1}{}
    \depedge[red]{3}{5}{}
    \depedge[dashed,ultra thick]{7}{5}{}
    \deproot[edge height=0.7cm, ultra thick]{9}{}
    \wordgroup{1}{3}{3}{}
    \wordgroup{1}{5}{5}{}
\end{dependency}
\end{figure}
    \begin{itemize}
        \item \textbf{S}hift: \emph{Push} the first word of the buffer onto the stack
    \item \textbf{R}ight-Arc: Set top-of-stack as Head of start-of-buffer + \emph{Push}.
    \item Re\textbf{D}uce: \emph{Pop} the top word off the stack
    \item \textbf{L}eft-Arc: Set start-of-buffer as Head of top-of-stack + \emph{Pop}.
    \end{itemize}
\end{plain}

\begin{plain}{Arc-Eager move constraints}
Every word is pushed once, popped once, and has its head set once:
\begin{table}
    \centering
    \begin{tabular}{l|ll}
           & Push           & Pop \\
        \hline
Adds head  & \emph{Right-Arc}   & \textbf{Left-Arc}    \\
No head    & \textbf{Shift} & \emph{Reduce}    \\
\end{tabular}
\end{table}
To ensure one head per word, the pop moves are constrained:
\begin{itemize}
    \item \textbf{Left-Arc} only if top-of-stack was pushed via Shift (no head);
    \item \textbf{Reduce} only if top-of-stack was pushed via Right-Arc (has head).
\end{itemize}
We remove these constraints, and let the parser revise the previous Shift/Right-Arc
decision if the model recommends it.
\end{plain}

\begin{plain}{State needing Non-monotonic Left-Arc}


\begin{figure}
    \centering
    \begin{dependency}[theme=simple]
        \tikzstyle{t}=[text=green,ultra thick,font=\bfseries\itshape]
        \tikzstyle{f}=[text=red,ultra thick,font=\bfseries\itshape]
        \tikzstyle{m}=[font=\bfseries\itshape]
        \tikzstyle{n}=[font=\itshape]

        \begin{deptext}[column sep=.075cm, row sep=.1ex]
            I \&           \& saw \&          \& Jack       \& \& and     \&           \& Jill   \&  \& fall \& \textsc{r} \\
|[m]|S \& |[m]|L \& |[m]|S   \& |[f]|R \& |[m]|R \& |[m]|D \& |[m]|R \& |[m]|D \& \\

    1 \&     2       \& 3  \&   4      \& 5          \& 6 \& 7     \& 8 \& 9 \& 10 \& 11 \& 12 \\
    I \&           \& saw \&          \& Jack       \& \& and     \&           \& Jill   \&      \& fall \& \textsc{r} \\
        \end{deptext}
    \depedge{3}{1}{2}
    \depedge[red, ultra thick]{3}{5}{4}
    \depedge{5}{7}{5}
    \depedge{5}{9}{7}
    
    \deproot[edge height=0.7cm, ultra thick]{11}{}
    \wordgroup{1}{3}{3}{}
    \wordgroup{1}{5}{5}{}
\end{dependency}
\end{figure}
\end{plain}


\begin{plain}{Before (above) and After (below) Non-monotonic Left-Arc}


\begin{figure}
    \centering
    \begin{dependency}[theme=simple]
        \tikzstyle{t}=[text=green,ultra thick,font=\bfseries\itshape]
        \tikzstyle{f}=[text=red,ultra thick,font=\bfseries\itshape]
        \tikzstyle{m}=[font=\bfseries\itshape]
        \tikzstyle{n}=[font=\itshape]

        \begin{deptext}[column sep=.075cm, row sep=.1ex]
            I \&           \& saw \&          \& Jack       \& \& and     \&           \& Jill   \&  \& fall \& \textsc{r} \\
|[m]|S \& |[m]|L \& |[m]|S   \& |[f]|R \& |[m]|R \& |[m]|D \& |[m]|R \& |[m]|D \& |[t]|NM-L \\
    1 \&     2       \& 3  \&   4      \& 5          \& 6 \& 7     \& 8 \& 9 \& 10 \& 11 \& 12 \\
    I \&           \& saw \&          \& Jack       \& \& and     \&           \& Jill   \&      \& fall \& \textsc{r} \\
        \end{deptext}
    \depedge{3}{1}{2}
    \depedge[red, ultra thick]{3}{5}{4}
    \depedge{5}{7}{5}
    \depedge{5}{9}{7}
    
    \deproot[edge height=0.7cm, ultra thick]{11}{}
    \wordgroup{1}{3}{3}{}
    \wordgroup{1}{5}{5}{}
    
    \depedge[edge below]{3}{1}{2}
    \depedge[edge below]{5}{7}{5}
    \depedge[edge below]{5}{9}{7}
    \depedge[edge below, green, ultra thick]{11}{5}{9}
\end{dependency}
\end{figure}

We allow the Left-Arc to `clobber' the existing head of \emph{Jack},  \emph{saw}.
It sets \emph{fall} as the head of \emph{Jack} instead. 

\end{plain}

\begin{plain}{State needing Non-monotonic Reduce}
\begin{figure}
    \centering
    \begin{dependency}[theme=simple]
    \tikzstyle{t}=[text=green,ultra thick,font=\bfseries\itshape]
    \tikzstyle{f}=[text=red,ultra thick,font=\bfseries\itshape]
    \tikzstyle{m}=[font=\bfseries\itshape]
    \tikzstyle{n}=[font=\itshape]
    \begin{deptext}[column sep=.075cm, row sep=.1ex]
        I \&           \& saw \&          \& Jack       \& \& and     \&           \& Jill \&   \& \textsc{r} \\
       |[m]|S \& |[m]|L \& |[m]|S   \& |[f]|S \& |[m]|R \& |[m]|D \& |[m]|R \& |[m]|D  \\
1 \&     2       \& 3  \&   4      \& 5          \& 6 \& 7     \& 8 \\
\end{deptext}
    \depedge{3}{1}{2}
    \depedge{5}{7}{5}
    \depedge{5}{9}{7}
    
    \deproot[edge height=0.7cm, ultra thick]{11}{}
    \wordgroup{1}{3}{3}{}
    \wordgroup{1}{5}{5}{}
\end{dependency}
\end{figure}
The Move 4 Shift leaves \emph{Jack} waiting for a head from the buffer.
\end{plain}


\begin{plain}{Before (above) and After (below) Non-monotonic Reduce}
\begin{figure}
    \centering
    \begin{dependency}[theme=simple]
    \tikzstyle{t}=[text=green,ultra thick,font=\bfseries\itshape]
    \tikzstyle{f}=[text=red,ultra thick,font=\bfseries\itshape]
    \tikzstyle{m}=[font=\bfseries\itshape]
    \tikzstyle{n}=[font=\itshape]
    \begin{deptext}[column sep=.075cm, row sep=.1ex]
        I \&           \& saw \&          \& Jack       \& \& and     \&           \& Jill \&   \& \textsc{r} \\
        |[m]|S \& |[m]|L \& |[m]|S   \& |[f]|S \& |[m]|R \& |[m]|D \& |[m]|R \& |[m]|D \& |[t]|NM-D \\
        1 \&     2       \& 3  \&   4      \& 5          \& 6 \& 7     \& 8 \& 9 \\

        I \&           \& saw \&          \& Jack       \& \& and     \&           \& Jill \& \& \textsc{r} \\
\end{deptext}
    \depedge{3}{1}{2}
    \depedge{5}{7}{5}
    \depedge{5}{9}{7}
    
    \deproot[edge height=0.7cm, ultra thick]{11}{}
    \wordgroup{1}{3}{3}{}
    \wordgroup{1}{5}{5}{}
    
    \depedge[edge below]{3}{1}{2}
    \depedge[edge below]{5}{7}{5}
    \depedge[edge below]{5}{9}{7}
    \depedge[edge below, green, ultra thick]{3}{5}{9}
    \wordgroup{4}{3}{3}{}
\end{dependency}
\end{figure}
Non-monotonic Reduce pops \emph{Jack}, and an
arc is inserted from \emph{saw} --- as though Right-Arc were selected at move 4.
\end{plain}

\begin{points}{Dynamic oracle used for training}
\p Need examples of mistakes to be corrected
\p Until recently, we only knew how to map a dependency parse to a single,
gold-standard transition sequence
\p Goldberg and Nivre's (2012) \emph{dynamic oracle} for training: \\
\begin{enumerate}
    \item Use model to score all actions
    \item Use oracle to ascribe a \emph{cost} to each possible action, for how many
          arcs it makes newly unreachable
    \item Label the example with the highest-scoring zero-cost action
    \item Apply highest-scoring action --- \emph{even if non-zero cost}
    \item Continue training from the resulting state
\end{enumerate}
\p Goldberg and Nivre (2012) is our baseline system (previous state-of-the-art)
for greedy deterministic parsing
\end{points}

\begin{plain}{WSJ test accuracies}
\begin{table}
%UAS
%baseline 21: 90.9 +/- 0.04
%both 21: 91.1 +/- 0.07
%LAS
%baseline 21: 88.7 +/- 0.05
%both 21: 88.9 +/- 0.07
% UAS
% baseline 21: 90.6 +/- 0.07
% both 21: 91.0 +/- 0.05
% LAS
% baseline 21: 89.5 +/- 0.07
% both 21: 89.9 +/- 0.06
    \centering
    \small
    \begin{tabular}{l|rr|rr}
        \hline 
System  &    \multicolumn{2}{c}{Stanford} & \multicolumn{2}{|c}{Penn2Malt} \\
                    &  LAS  & UAS  & LAS & UAS \\
        \hline \hline
Baseline(G\&N-12)   & 88.7 & 90.9 & 88.7  & 90.6 \\
NM L+D              & 88.9 & 91.1 & 88.9  & 91.0 \\
\hline
    \end{tabular}
\end{table}

Also: statistically significant improvement on 5/10 CoNLL languages; 3/10 trending up;
none decline.
\end{plain}


\begin{points}{When do they help?}
    \p Subordinating conjunction vs. preposition\\
    \emph{Stores were closed \textbf{after} shoppers left}\\
    Here \emph{after} is headed by \emph{left}\\
    vs.\\
    \emph{Stores were closed \textbf{after} midnight}\\
    Here \emph{after} is headed by \emph{closed}
    \p Direct object vs. complement clause\\
    \emph{NCNB corp. reported net \textbf{income} last night}\\
    Here \emph{income} is headed by \emph{reported}\\
    vs\\
    \emph{NCNB corp. reported net \textbf{income} doubled}\\
    Here \emph{income} is headed by \emph{doubled}\\
\end{points}

\begin{plain}{How often are non-monotonic moves used on dev. data?}
\begin{table}
    \centering
    \small
    \begin{tabular}{l|rrrr}
\hline
         & TP  &  FP & TN     & FN \\
\hline \hline
Left-Arc & 60  & 14  & 18,466 & 285 \\
Reduce   & 52  & 26  & 14,066 & 250  \\
Total    & 112 & 40  & 32,532  & 535 \\
\hline
\end{tabular}
\caption{\small
    True/False positive/negative rates for the prediction of the non-monotonic
    transitions. The non-monotonic transitions add
    correct dependencies 112 times, and produce worse parses 40 times.
    535 opportunities for non-monotonic transitions were missed.
\label{tab:confusions}}
\end{table}
\end{plain}

\begin{points}{Conclusion and future work}
    \p Monotonicity has been attractive, as it seems to make search simpler
       (less spurious ambiguity)
    \p But if we're not searching much of the space anyway, it may be worth
       relaxing it
       \vspace{0.2in}
    \p Limited non-monotonic behaviour improves a state-of-the-art greedy
    deterministic parser by 0.2\% UAS on English (p $< 0.001$).
    \p Improvement robust across different languages and labelling schemes.
    \vspace{0.2in}
    \p Preliminary results suggest non-monotonic behaviour will be useful for
       parsing speech disfluencies.
\end{points}

\end{document}
