
\documentclass{lecture}
\DeclareMathOperator*{\argmax}{arg\,max}
\DeclareMathOperator*{\argmin}{arg\,min}

\usepackage{algorithmicx}

\usepackage{tikz-dependency}
\usepackage{placeins}
\usepackage{xcolor}
\usepackage{cancel}

\title{Incremental parsing of disfluent speech}
\author{Matthew Honnibal}


\begin{document}

\titleslide

\begin{points}{Parsing, uh, incremental parsing of disfluencies}
\p Spontaneous speech often contains disfluencies:\\
\emph{I want a flight to Boston, uh, I mean, to Denver on Friday}
\p Fluent transcripts are better for downstream tasks, e.g. information extraction
\p Disfluencies are also a problem for inter-mediate processes, e.g. parsing
\p (Most) Previous work: \emph{Pipeline} approach. First detect disfluencies, then parse
\p This work: \emph{Joint} approach: Detect disfluencies during parsing.
\end{points}

\begin{plain}{Why incremental dependency parsing?}
    \begin{enumerate}
        \item No grammar --- almost entirely `empiricist'\\
            \begin{itemize}
                \item \textsc{pcfg}s encode some small linguistic insights, e.g.
                      they expect to learn recursive structures
                  \item What we're trying to learn (disfluent trees) aren't
                        really grammatical sentences,
                      so those assumptions might be costly
            \end{itemize}
        \item Linear time, so won't need sentence segmentation
        \item Results are surprisingly strong. On WSJ:\\
            \begin{table}
            \begin{tabular}{l|rr}
                \hline
                Model         & UAS   & Sents/Second \\
                \hline \hline
                Stanford PCFG & 90.1 & 3.5 \\
                Stanford Factored & 91.8 & 0.9 \\
                Redshift Greedy & 91.6 & 865 \\
                Redshift Beam=8 & 93.2  & 212 \\
                Redshift beam=64 & 93.5 & 34 \\
                \hline
            \end{tabular}
        \end{table}
    \end{enumerate}
\end{plain}

\begin{plain}{Arc-Eager Incremental Dependency Parsing}
\begin{figure}
    \centering
    \begin{dependency}[theme=simple]
    \tikzstyle{t}=[text=green,ultra thick,font=\bfseries\itshape]
    \tikzstyle{q}=[text=gray,very thin, dashed]
    \tikzstyle{m}=[font=\bfseries\itshape]
    \tikzstyle{n}=[font=\itshape]
    \begin{deptext}[column sep=.075cm, row sep=.1ex]
        I \& |[q]|'d \& |[q]|like \\
         \& \\
    \end{deptext}
    \deproot[edge height=0.7cm, ultra thick]{1}{}
\end{dependency}
\end{figure}
    \begin{itemize}
        \item \textbf{S}hift: \emph{Push} the first word of the buffer onto the stack
    \item \textbf{R}ight-Arc: Set top-of-stack as Head of start-of-buffer + \emph{Push}.
    \item Re\textbf{D}uce: \emph{Pop} the top word off the stack
    \item \textbf{L}eft-Arc: Set start-of-buffer as Head of top-of-stack + \emph{Pop}.
    \end{itemize}
\end{plain}


\begin{plain}{Stack circled, arrow at buffer, look-ahead grey}
\begin{figure}
    \centering
    \begin{dependency}[theme=simple]
    \tikzstyle{t}=[text=green,ultra thick,font=\bfseries\itshape]
    \tikzstyle{q}=[text=gray,very thin, dashed]
    \tikzstyle{m}=[font=\bfseries\itshape]
    \tikzstyle{n}=[font=\itshape]
    \begin{deptext}[column sep=.075cm, row sep=.1ex]
        I \& 'd \& |[q]|like \& |[q]|a \\
        \& \\
    \end{deptext}
    \deproot[edge height=0.7cm, ultra thick]{2}{}
    \wordgroup{1}{1}{1}{}
\end{dependency}
\end{figure}
    \begin{itemize}
        \item \textbf{S}hift: \emph{Push} the first word of the buffer onto the stack
    \item \textbf{R}ight-Arc: Set top-of-stack as Head of start-of-buffer + \emph{Push}.
    \item Re\textbf{D}uce: \emph{Pop} the top word off the stack
    \item \textbf{L}eft-Arc: Set start-of-buffer as Head of top-of-stack + \emph{Pop}.
    \end{itemize}
\end{plain}

\begin{plain}{Stack circled, arrow at buffer, look-ahead grey}
\begin{figure}
    \centering
    \begin{dependency}[theme=simple]
    \tikzstyle{t}=[text=green,ultra thick,font=\bfseries\itshape]
    \tikzstyle{q}=[text=gray,very thin, dashed]
    \tikzstyle{m}=[font=\bfseries\itshape]
    \tikzstyle{n}=[font=\itshape]
    \begin{deptext}[column sep=.075cm, row sep=.1ex]
        I \& 'd \& like \& |[q]|a \& |[q]|flight \\
         \& \\
    \end{deptext}
    \deproot[edge height=0.7cm, ultra thick]{3}{}
    %\depedge{2}{1}{}
    \wordgroup{1}{1}{1}{}
    \wordgroup{1}{2}{2}{}
\end{dependency}
\end{figure}
    \begin{itemize}
        \item \textbf{S}hift: \emph{Push} the first word of the buffer onto the stack
    \item \textbf{R}ight-Arc: Set top-of-stack as Head of start-of-buffer + \emph{Push}.
    \item Re\textbf{D}uce: \emph{Pop} the top word off the stack
    \item \textbf{L}eft-Arc: Set start-of-buffer as Head of top-of-stack + \emph{Pop}.
    \end{itemize}
\end{plain}


\begin{plain}{Stack circled, arrow at buffer, look-ahead grey}
\begin{figure}
    \centering
    \begin{dependency}[theme=simple]
    \tikzstyle{t}=[text=green,ultra thick,font=\bfseries\itshape]
    \tikzstyle{q}=[text=gray,very thin, dashed]
    \tikzstyle{m}=[font=\bfseries\itshape]
    \tikzstyle{n}=[font=\itshape]
    \begin{deptext}[column sep=.075cm, row sep=.1ex]
        I \& 'd \& like \& |[q]|a \& |[q]|flight \\
         \& \\
    \end{deptext}
    \deproot[edge height=0.7cm, ultra thick]{3}{}
    \depedge{3}{2}{}
    \wordgroup{1}{1}{1}{}
\end{dependency}
\end{figure}
    \begin{itemize}
        \item \textbf{S}hift: \emph{Push} the first word of the buffer onto the stack
    \item \textbf{R}ight-Arc: Set top-of-stack as Head of start-of-buffer + \emph{Push}.
    \item Re\textbf{D}uce: \emph{Pop} the top word off the stack
    \item \textbf{L}eft-Arc: Set start-of-buffer as Head of top-of-stack + \emph{Pop}.
    \end{itemize}
\end{plain}


\begin{plain}{Stack circled, arrow at buffer, look-ahead grey}
\begin{figure}
    \centering
    \begin{dependency}[theme=simple]
    \tikzstyle{t}=[text=green,ultra thick,font=\bfseries\itshape]
    \tikzstyle{q}=[text=gray,very thin, dashed]
    \tikzstyle{m}=[font=\bfseries\itshape]
    \tikzstyle{n}=[font=\itshape]
    \begin{deptext}[column sep=.075cm, row sep=.1ex]
        I \& 'd \& like \& |[q]|a \& |[q]|flight \\
         \& \\
    \end{deptext}
    \deproot[edge height=0.7cm, ultra thick]{3}{}
    \depedge{3}{2}{}
    \depedge{3}{1}{}
\end{dependency}
\end{figure}
    \begin{itemize}
        \item \textbf{S}hift: \emph{Push} the first word of the buffer onto the stack
    \item \textbf{R}ight-Arc: Set top-of-stack as Head of start-of-buffer + \emph{Push}.
    \item Re\textbf{D}uce: \emph{Pop} the top word off the stack
    \item \textbf{L}eft-Arc: Set start-of-buffer as Head of top-of-stack + \emph{Pop}.
    \end{itemize}
\end{plain}


\begin{plain}{Stack circled, arrow at buffer, look-ahead grey}
\begin{figure}
    \centering
    \begin{dependency}[theme=simple]
    \tikzstyle{t}=[text=green,ultra thick,font=\bfseries\itshape]
    \tikzstyle{q}=[text=gray,very thin, dashed]
    \tikzstyle{m}=[font=\bfseries\itshape]
    \tikzstyle{n}=[font=\itshape]
    \begin{deptext}[column sep=.075cm, row sep=.1ex]
        I \& 'd \& like \& a \& |[q]|flight \& |[q]|to \\
         \& \\
    \end{deptext}
    \deproot[edge height=0.7cm, ultra thick]{4}{}
    \depedge{3}{2}{}
    \depedge{3}{1}{}
    \wordgroup{1}{3}{3}{}
\end{dependency}
\end{figure}
    \begin{itemize}
        \item \textbf{S}hift: \emph{Push} the first word of the buffer onto the stack
    \item \textbf{R}ight-Arc: Set top-of-stack as Head of start-of-buffer + \emph{Push}.
    \item Re\textbf{D}uce: \emph{Pop} the top word off the stack
    \item \textbf{L}eft-Arc: Set start-of-buffer as Head of top-of-stack + \emph{Pop}.
    \end{itemize}
\end{plain}



\begin{plain}{Stack circled, arrow at buffer, look-ahead grey}
\begin{figure}
    \centering
    \begin{dependency}[theme=simple]
    \tikzstyle{t}=[text=green,ultra thick,font=\bfseries\itshape]
    \tikzstyle{q}=[text=gray,very thin, dashed]
    \tikzstyle{m}=[font=\bfseries\itshape]
    \tikzstyle{n}=[font=\itshape]
    \begin{deptext}[column sep=.075cm, row sep=.1ex]
        I \& 'd \& like \& a \& flight \& |[q]|to \& |[q]|Boston \\
         \& \\
    \end{deptext}
    \deproot[edge height=0.7cm, ultra thick]{5}{}
    \depedge{3}{2}{}
    \depedge{3}{1}{}
    \wordgroup{1}{3}{3}{}
    \wordgroup{1}{4}{4}{}
\end{dependency}
\end{figure}
    \begin{itemize}
        \item \textbf{S}hift: \emph{Push} the first word of the buffer onto the stack
    \item \textbf{R}ight-Arc: Set top-of-stack as Head of start-of-buffer + \emph{Push}.
    \item Re\textbf{D}uce: \emph{Pop} the top word off the stack
    \item \textbf{L}eft-Arc: Set start-of-buffer as Head of top-of-stack + \emph{Pop}.
    \end{itemize}
\end{plain}



\begin{plain}{Stack circled, arrow at buffer, look-ahead grey}
\begin{figure}
    \centering
    \begin{dependency}[theme=simple]
    \tikzstyle{t}=[text=green,ultra thick,font=\bfseries\itshape]
    \tikzstyle{q}=[text=gray,very thin, dashed]
    \tikzstyle{m}=[font=\bfseries\itshape]
    \tikzstyle{n}=[font=\itshape]
    \begin{deptext}[column sep=.075cm, row sep=.1ex]
        I \& 'd \& like \& a \& flight \& |[q]|to \& |[q]|Boston \\
         \& \\
    \end{deptext}
    \deproot[edge height=0.7cm, ultra thick]{5}{}
    \depedge{3}{2}{}
    \depedge{3}{1}{}
    \depedge{5}{4}{}
    \wordgroup{1}{3}{3}{}
\end{dependency}
\end{figure}
    \begin{itemize}
        \item \textbf{S}hift: \emph{Push} the first word of the buffer onto the stack
    \item \textbf{R}ight-Arc: Set top-of-stack as Head of start-of-buffer + \emph{Push}.
    \item Re\textbf{D}uce: \emph{Pop} the top word off the stack
    \item \textbf{L}eft-Arc: Set start-of-buffer as Head of top-of-stack + \emph{Pop}.
    \end{itemize}
\end{plain}


\begin{plain}{Stack circled, arrow at buffer, look-ahead grey}
\begin{figure}
    \centering
    \begin{dependency}[theme=simple]
    \tikzstyle{t}=[text=green,ultra thick,font=\bfseries\itshape]
    \tikzstyle{q}=[text=gray,very thin, dashed]
    \tikzstyle{m}=[font=\bfseries\itshape]
    \tikzstyle{n}=[font=\itshape]
    \begin{deptext}[column sep=.075cm, row sep=.1ex]
        I \& 'd \& like \& a \& flight \& to \& |[q]|Boston \& |[q]| uh \\
         \& \\
    \end{deptext}
    \deproot[edge height=0.7cm, ultra thick]{6}{}
    \depedge{3}{2}{}
    \depedge{3}{1}{}
    \depedge{5}{4}{}
    \depedge{3}{5}{}
    \wordgroup{1}{3}{3}{}
    \wordgroup{1}{5}{5}{}
\end{dependency}
\end{figure}
    \begin{itemize}
        \item \textbf{S}hift: \emph{Push} the first word of the buffer onto the stack
    \item \textbf{R}ight-Arc: Set top-of-stack as Head of start-of-buffer + \emph{Push}.
    \item Re\textbf{D}uce: \emph{Pop} the top word off the stack
    \item \textbf{L}eft-Arc: Set start-of-buffer as Head of top-of-stack + \emph{Pop}.
    \end{itemize}
\end{plain}


\begin{plain}{Stack circled, arrow at buffer, look-ahead grey}
\begin{figure}
    \centering
    \begin{dependency}[theme=simple]
    \tikzstyle{t}=[text=green,ultra thick,font=\bfseries\itshape]
    \tikzstyle{q}=[text=gray,very thin, dashed]
    \tikzstyle{m}=[font=\bfseries\itshape]
    \tikzstyle{n}=[font=\itshape]
    \begin{deptext}[column sep=.075cm, row sep=.1ex]
        I \& 'd \& like \& a \& flight \& to \& Boston \& |[q]| uh \& |[q]| Denver \\
         \& \\
    \end{deptext}
    \deproot[edge height=0.7cm, ultra thick]{7}{}
    \depedge{3}{2}{}
    \depedge{3}{1}{}
    \depedge{5}{4}{}
    \depedge{3}{5}{}
    \depedge{5}{6}{}
    \wordgroup{1}{3}{3}{}
    \wordgroup{1}{5}{5}{}
    \wordgroup{1}{6}{6}{}
\end{dependency}
\end{figure}
    \begin{itemize}
        \item \textbf{S}hift: \emph{Push} the first word of the buffer onto the stack
    \item \textbf{R}ight-Arc: Set top-of-stack as Head of start-of-buffer + \emph{Push}.
    \item Re\textbf{D}uce: \emph{Pop} the top word off the stack
    \item \textbf{L}eft-Arc: Set start-of-buffer as Head of top-of-stack + \emph{Pop}.
    \end{itemize}
\end{plain}


\begin{plain}{Stack circled, arrow at buffer, look-ahead grey}
\begin{figure}
    \centering
    \begin{dependency}[theme=simple]
    \tikzstyle{t}=[text=green,ultra thick,font=\bfseries\itshape]
    \tikzstyle{q}=[text=gray,very thin, dashed]
    \tikzstyle{m}=[font=\bfseries\itshape]
    \tikzstyle{n}=[font=\itshape]
    \begin{deptext}[column sep=.075cm, row sep=.1ex]
        I \& 'd \& like \& a \& flight \& to \& Boston \&  uh \& |[q]| Denver \& |[q]|\textsc{r} \\
    \end{deptext}
    \deproot[edge height=0.7cm, ultra thick]{8}{}
    \depedge{3}{2}{}
    \depedge{3}{1}{}
    \depedge{5}{4}{}
    \depedge{3}{5}{}
    \depedge{5}{6}{}
    \depedge[edge below]{6}{7}{disfl}
    \wordgroup{1}{3}{3}{}
    \wordgroup{1}{5}{5}{}
    \wordgroup{1}{6}{6}{}
    \wordgroup{1}{7}{7}{}
\end{dependency}
\end{figure}
    \begin{itemize}
        \item \textbf{S}hift: \emph{Push} the first word of the buffer onto the stack
    \item \textbf{R}ight-Arc: Set top-of-stack as Head of start-of-buffer + \emph{Push}.
    \item Re\textbf{D}uce: \emph{Pop} the top word off the stack
    \item \textbf{L}eft-Arc: Set start-of-buffer as Head of top-of-stack + \emph{Pop}.
    \end{itemize}
\end{plain}

\begin{plain}{Stack circled, arrow at buffer, look-ahead grey}
\begin{figure}
    \centering
    \begin{dependency}[theme=simple]
    \tikzstyle{t}=[text=green,ultra thick,font=\bfseries\itshape]
    \tikzstyle{q}=[text=gray,very thin, dashed]
    \tikzstyle{m}=[font=\bfseries\itshape]
    \tikzstyle{n}=[font=\itshape]
    \begin{deptext}[column sep=.075cm, row sep=.1ex]
        I \& 'd \& like \& a \& flight \& to \& Boston \&  uh \& |[q]| Denver \& |[q]|\textsc{r} \\
    \end{deptext}
    \deproot[edge height=0.7cm, ultra thick]{8}{}
    \depedge{3}{2}{}
    \depedge{3}{1}{}
    \depedge{5}{4}{}
    \depedge{3}{5}{}
    \depedge{5}{6}{}
    \depedge[edge below]{6}{7}{disfl}
    \wordgroup{1}{3}{3}{}
    \wordgroup{1}{5}{5}{}
    \wordgroup{1}{6}{6}{}
\end{dependency}
\end{figure}
    \begin{itemize}
        \item \textbf{S}hift: \emph{Push} the first word of the buffer onto the stack
    \item \textbf{R}ight-Arc: Set top-of-stack as Head of start-of-buffer + \emph{Push}.
    \item Re\textbf{D}uce: \emph{Pop} the top word off the stack
    \item \textbf{L}eft-Arc: Set start-of-buffer as Head of top-of-stack + \emph{Pop}.
    \end{itemize}
\end{plain}


\begin{plain}{Stack circled, arrow at buffer, look-ahead grey}
\begin{figure}
    \centering
    \begin{dependency}[theme=simple]
    \tikzstyle{t}=[text=green,ultra thick,font=\bfseries\itshape]
    \tikzstyle{q}=[text=gray,very thin, dashed]
    \tikzstyle{m}=[font=\bfseries\itshape]
    \tikzstyle{n}=[font=\itshape]
    \begin{deptext}[column sep=.075cm, row sep=.1ex]
        I \& 'd \& like \& a \& flight \& to \& Boston \& uh \& Denver \&  |[q]|\textsc{r} \\
    \end{deptext}
    \deproot[edge height=0.7cm, ultra thick]{9}{}
    \depedge{3}{2}{}
    \depedge{3}{1}{}
    \depedge{5}{4}{}
    \depedge{3}{5}{}
    \depedge{5}{6}{}
    \depedge[edge below]{6}{7}{erased}
    \depedge{6}{8}{IP}
    \wordgroup{1}{3}{3}{}
    \wordgroup{1}{5}{5}{}
    \wordgroup{1}{6}{6}{}
    \wordgroup{1}{8}{8}{}
\end{dependency}
\end{figure}
    \begin{itemize}
        \item \textbf{S}hift: \emph{Push} the first word of the buffer onto the stack
    \item \textbf{R}ight-Arc: Set top-of-stack as Head of start-of-buffer + \emph{Push}.
    \item Re\textbf{D}uce: \emph{Pop} the top word off the stack
    \item \textbf{L}eft-Arc: Set start-of-buffer as Head of top-of-stack + \emph{Pop}.
    \end{itemize}
\end{plain}


\begin{plain}{Stack circled, arrow at buffer, look-ahead grey}
\begin{figure}
    \centering
    \begin{dependency}[theme=simple]
    \tikzstyle{t}=[text=green,ultra thick,font=\bfseries\itshape]
    \tikzstyle{q}=[text=gray,very thin, dashed]
    \tikzstyle{m}=[font=\bfseries\itshape]
    \tikzstyle{n}=[font=\itshape]
    \begin{deptext}[column sep=.075cm, row sep=.1ex]
        I \& 'd \& like \& a \& flight \& to \& Boston \& uh \& Denver \&  |[q]|\textsc{r} \\
    \end{deptext}
    \deproot[edge height=0.7cm, ultra thick]{9}{}
    \depedge{3}{2}{}
    \depedge{3}{1}{}
    \depedge{5}{4}{}
    \depedge{3}{5}{}
    \depedge{5}{6}{}
    \depedge[edge below]{6}{7}{erased}
    \depedge{6}{8}{IP}
    \wordgroup{1}{3}{3}{}
    \wordgroup{1}{5}{5}{}
    \wordgroup{1}{6}{6}{}
\end{dependency}
\end{figure}
    \begin{itemize}
        \item \textbf{S}hift: \emph{Push} the first word of the buffer onto the stack
    \item \textbf{R}ight-Arc: Set top-of-stack as Head of start-of-buffer + \emph{Push}.
    \item Re\textbf{D}uce: \emph{Pop} the top word off the stack
    \item \textbf{L}eft-Arc: Set start-of-buffer as Head of top-of-stack + \emph{Pop}.
    \end{itemize}
\end{plain}

\begin{plain}{Stack circled, arrow at buffer, look-ahead grey}
\begin{figure}
    \centering
    \begin{dependency}[theme=simple]
    \tikzstyle{t}=[text=green,ultra thick,font=\bfseries\itshape]
    \tikzstyle{q}=[text=gray,very thin, dashed]
    \tikzstyle{m}=[font=\bfseries\itshape]
    \tikzstyle{n}=[font=\itshape]
    \begin{deptext}[column sep=.075cm, row sep=.1ex]
        I \& 'd \& like \& a \& flight \& to \& Boston \& uh \& Denver \&  \textsc{r} \\
    \end{deptext}
    \deproot[edge height=0.7cm, ultra thick]{10}{}
    \depedge{3}{2}{}
    \depedge{3}{1}{}
    \depedge{5}{4}{}
    \depedge{3}{5}{}
    \depedge{5}{6}{}
    \depedge[edge below]{6}{7}{erased}
    \depedge{6}{8}{IP}
    \depedge{6}{9}{}
    \wordgroup{1}{3}{3}{}
    \wordgroup{1}{5}{5}{}
    \wordgroup{1}{6}{6}{}
    \wordgroup{1}{9}{9}{}
\end{dependency}

\end{figure}
    \begin{itemize}
        \item \textbf{S}hift: \emph{Push} the first word of the buffer onto the stack
    \item \textbf{R}ight-Arc: Set top-of-stack as Head of start-of-buffer + \emph{Push}.
    \item Re\textbf{D}uce: \emph{Pop} the top word off the stack
    \item \textbf{L}eft-Arc: Set start-of-buffer as Head of top-of-stack + \emph{Pop}.
    \end{itemize}
\end{plain}


\begin{plain}{Stack circled, arrow at buffer, look-ahead grey}
\begin{figure}
    \centering
    \begin{dependency}[theme=simple]
    \tikzstyle{t}=[text=green,ultra thick,font=\bfseries\itshape]
    \tikzstyle{q}=[text=gray,very thin, dashed]
    \tikzstyle{m}=[font=\bfseries\itshape]
    \tikzstyle{n}=[font=\itshape]
    \begin{deptext}[column sep=.075cm, row sep=.1ex]
        I \& 'd \& like \& a \& flight \& to \& Boston \& uh \& Denver \&  \textsc{r} \\
    \end{deptext}
    \deproot[edge height=0.7cm, ultra thick]{10}{}
    \depedge{3}{2}{}
    \depedge{3}{1}{}
    \depedge{5}{4}{}
    \depedge{3}{5}{}
    \depedge{5}{6}{}
    \depedge[edge below]{6}{7}{erased}
    \depedge{6}{8}{IP}
    \depedge{6}{9}{}
    \wordgroup{1}{3}{3}{}
    \wordgroup{1}{5}{5}{}
    \wordgroup{1}{6}{6}{}
\end{dependency}

\end{figure}
    \begin{itemize}
        \item \textbf{S}hift: \emph{Push} the first word of the buffer onto the stack
    \item \textbf{R}ight-Arc: Set top-of-stack as Head of start-of-buffer + \emph{Push}.
    \item Re\textbf{D}uce: \emph{Pop} the top word off the stack
    \item \textbf{L}eft-Arc: Set start-of-buffer as Head of top-of-stack + \emph{Pop}.
    \end{itemize}
\end{plain}



\begin{plain}{Stack circled, arrow at buffer, look-ahead grey}
\begin{figure}
    \centering
    \begin{dependency}[theme=simple]
    \tikzstyle{t}=[text=green,ultra thick,font=\bfseries\itshape]
    \tikzstyle{q}=[text=gray,very thin, dashed]
    \tikzstyle{m}=[font=\bfseries\itshape]
    \tikzstyle{n}=[font=\itshape]
    \begin{deptext}[column sep=.075cm, row sep=.1ex]
        I \& 'd \& like \& a \& flight \& to \& Boston \& uh \& Denver \&  \textsc{r} \\
    \end{deptext}
    \deproot[edge height=0.7cm, ultra thick]{10}{}
    \depedge{3}{2}{}
    \depedge{3}{1}{}
    \depedge{5}{4}{}
    \depedge{3}{5}{}
    \depedge{5}{6}{}
    \depedge[edge below]{6}{7}{erased}
    \depedge{6}{8}{IP}
    \depedge{6}{9}{}
    \wordgroup{1}{3}{3}{}
    \wordgroup{1}{5}{5}{}
\end{dependency}

\end{figure}
    \begin{itemize}
        \item \textbf{S}hift: \emph{Push} the first word of the buffer onto the stack
    \item \textbf{R}ight-Arc: Set top-of-stack as Head of start-of-buffer + \emph{Push}.
    \item Re\textbf{D}uce: \emph{Pop} the top word off the stack
    \item \textbf{L}eft-Arc: Set start-of-buffer as Head of top-of-stack + \emph{Pop}.
    \end{itemize}
\end{plain}



\begin{plain}{Stack circled, arrow at buffer, look-ahead grey}
\begin{figure}
    \centering
    \begin{dependency}[theme=simple]
    \tikzstyle{t}=[text=green,ultra thick,font=\bfseries\itshape]
    \tikzstyle{q}=[text=gray,very thin, dashed]
    \tikzstyle{m}=[font=\bfseries\itshape]
    \tikzstyle{n}=[font=\itshape]
    \begin{deptext}[column sep=.075cm, row sep=.1ex]
        I \& 'd \& like \& a \& flight \& to \& Boston \& uh \& Denver \&  \textsc{r} \\
    \end{deptext}
    \deproot[edge height=0.7cm, ultra thick]{10}{}
    \depedge{3}{2}{}
    \depedge{3}{1}{}
    \depedge{5}{4}{}
    \depedge{3}{5}{}
    \depedge{5}{6}{}
    \depedge[edge below]{6}{7}{erased}
    \depedge{6}{8}{IP}
    \depedge{6}{9}{}
    \wordgroup{1}{3}{3}{}
\end{dependency}

\end{figure}
    \begin{itemize}
        \item \textbf{S}hift: \emph{Push} the first word of the buffer onto the stack
    \item \textbf{R}ight-Arc: Set top-of-stack as Head of start-of-buffer + \emph{Push}.
    \item Re\textbf{D}uce: \emph{Pop} the top word off the stack
    \item \textbf{L}eft-Arc: Set start-of-buffer as Head of top-of-stack + \emph{Pop}.
    \end{itemize}
\end{plain}



\begin{plain}{Stack circled, arrow at buffer, look-ahead grey}
\begin{figure}
    \centering
    \begin{dependency}[theme=simple]
    \tikzstyle{t}=[text=green,ultra thick,font=\bfseries\itshape]
    \tikzstyle{q}=[text=gray,very thin, dashed]
    \tikzstyle{m}=[font=\bfseries\itshape]
    \tikzstyle{n}=[font=\itshape]
    \begin{deptext}[column sep=.075cm, row sep=.1ex]
        I \& 'd \& like \& a \& flight \& to \& Boston \& uh \& Denver \&  \textsc{r} \\
    \end{deptext}
    \deproot[edge height=0.7cm, ultra thick]{10}{}
    \depedge{3}{2}{}
    \depedge{3}{1}{}
    \depedge{5}{4}{}
    \depedge{3}{5}{}
    \depedge{5}{6}{}
    \depedge[edge below]{6}{7}{erased}
    \depedge{6}{8}{IP}
    \depedge{6}{9}{}
    \depedge{10}{3}{ROOT}
\end{dependency}
\end{figure}
    \begin{itemize}
        \item \textbf{S}hift: \emph{Push} the first word of the buffer onto the stack
    \item \textbf{R}ight-Arc: Set top-of-stack as Head of start-of-buffer + \emph{Push}.
    \item Re\textbf{D}uce: \emph{Pop} the top word off the stack
    \item \textbf{L}eft-Arc: Set start-of-buffer as Head of top-of-stack + \emph{Pop}.
    \end{itemize}
\end{plain}


\begin{points}{Approach 1: JustParse}
    \p A disfluent sentence is a fluent sentence plus some other text
    \p We want a projective tree over the fluent sentence
    \p We don't care about the structure of the other text --- \textbf{so long as we mark it
    as an edit} for future reference.
    \p Let's learn a tree where the other text is included, and just label the
       disfluent regions differently
    \p Also, label `interruption points' differently to help the edit detection
\end{points}

\begin{points}{Extra Features 1: Match Features}
\p Limited number of \emph{context tokens} for top-of-stack,
    its parent and children, first-of-buffer, its children, next two tokens, etc.
\p For each pair of those tokens, ask:\\
\begin{itemize}
        \p Do the tokens' words match? e.g. S0$_{L0}$-N0$_{L0}$-wmatch
        \p Do the tokens' POS tags match? e.g. S0$_{L2}$-S0h-pmatch
        \p If the words match, what's the word? e.g. S0$_{L0}$-N0$_{L0}$-wmatch-the
        \p If the POS tags match, what's the tag? e.g. S0-N1-pmatch-DT
\end{itemize}
\p 18 tokens, so ${18 \choose 2} = 153$ combinations
\p $153 + 152 + (153 * v) + (153 * t)$ possible features added, for a vocabulary
   of size $v$ and a tag-set of size $t$
\end{points}

\begin{points}{Extra Features 2: Rough Copy}
    \p I extended the state representation to track the \emph{leftmost edge} of
       each word's subtree
    \p This lets me ask how much of the top of the stack and the start of the buffer
       string-match, regardless of how they're structured
    \p Feature for length of match, and also whether whole constituent matches
    \p Same features for POS match, instead of words
\end{points}

\begin{points}{Extra Features 3: History features}

\p Is the previous word edited?
\p Are the previous two words edited?
\p If the previous word's an edit, does it word-match the current word?
\p If the previous word's an edit, does it POS-match the current word?
\p If the previous word's an edit, what's its word?
\p If the previous word's an edit, what's its POS?
\p If the previous word's an edit, how much of S0  and N0's subtrees word-match?
\p If the previous 2 words are edits, how much of S0  and N0's subtrees word-match?
\p If the previous word's an edit, how much of S0  and N0's subtrees POS-match?
\p If the previous 2 words are edits, how much of S0  and N0's subtrees POS-match?

\end{points}

\begin{plain}{Parsing Results}
    \begin{table}
    \begin{tabular}{l|rr}
        \hline
        Model      & UAS   & LAS \\
        \hline \hline
        JustParse   & 90.8 & 88.6    \\
        +Feats      & 91.1 & 88.9    \\
        %EDIT+Feats & 91.4 & 89.2    \\
        %\hline
        %Oracle Edits & 92.5  & 90.1 \\
        \hline
    \end{tabular}
    \centering
\end{table}
\begin{itemize}
    \item Gold-standard POS tags used
    \item Gold-standard sentence/segment boundaries
    \item Arcs from disfluent words not scored
    \item Arcs to disfluent words from non-disfluent words penalised
    \item Arcs to words incorrectly labelled disfluent penalised
    \item Arcs from fillers (e.g. um, uh) not scored
\end{itemize}
\end{plain}

\begin{plain}{Disfluency detection results}
    \begin{table}
    \begin{tabular}{l|rrr}
        \hline
        Model         & P    & R     & F \\
        \hline \hline
        JustParse     & 80.2     & 65.0  & 71.8  \\
        +Feats        & 87.1 & 73.8  & 79.9 \\
        %EDIT+Feats    & 91.7 & 74.8  & 82.4 \\
        \hline
    \end{tabular}
    \centering
\end{table}
\end{plain}

\begin{points}{Problems with the JustParse model}
\p The model is trained to care as much about edit-internal structure as any other
   part of the tree
\p ...But we don't care about this structure
\p ...And it may not even be very meaningful
\p Edit regions are assigned whatever structure the Stanford dependency converter
   produces given the gold-standard PTB trees
\p This structure is unreliable for practical and theoretical reasons
\p \textbf{The truth is not necessarily a tree here}
\end{points}

\begin{plain}{Example of cyclic dependencies}
\begin{figure}
    \centering
    \begin{dependency}[theme=simple]
    \tikzstyle{t}=[text=green,ultra thick,font=\bfseries\itshape]
    \tikzstyle{q}=[text=gray,very thin, dashed]
    \tikzstyle{m}=[font=\bfseries\itshape]
    \tikzstyle{n}=[font=\itshape]
    \begin{deptext}[column sep=.075cm, row sep=.1ex]
        I \& want \& uh \& would \& like \\
    \end{deptext}
    \depedge{5}{4}{}
    \depedge[green]{5}{1}{}
    \depedge{4}{3}{IP}
    \depedge[red, edge below]{2}{1}{}
    \depedge{5}{2}{disfl}
\end{dependency}
\end{figure}
The arc in red \emph{does} make sense here. But ultimately we want the arc in green.
\end{plain}

\begin{points}{Learning latent structure for edit regions}
\p We have two problems:
\begin{itemize}
    \item We don't just want to trust the Stanford-converter's output;
    \item We want to accomodate these `helical' dependency structures
\end{itemize}
\p The first step to both is to let the structure of edited regions remain \emph{implicit}
\p That is, we don't want to force a weight update for two hypotheses about how
to structure an edit region
\p (But we do want to be strict with the model about the edit region boundaries)   
\end{points}

\begin{points}{Training a beam parser with a dynamic oracle}
    \p Standard training method:\\
\begin{enumerate}
    \item Map the gold tree to a single gold-standard transition sequence
    \item Score the gold-standard transitions
    \item Beam-search using the current weights for the best prediction
    \item If the best is not the gold-standard, do a weight update
\end{enumerate}
\p Dynamic oracle:
\begin{enumerate}
    \item Define an oracle that computes a \emph{cost} for each transition from the current
   state. May be multiple zero-cost transitions, thus multiple gold-standard
   transition sequences.
    \item Do beam search for the best predicted transition sequence 
    \item If zero-cost (i.e. gold-standard), no weight update
    \item Else, do beam search for the best gold-standard transition sequence
    \item Do weight update
\end{enumerate}
\p First use of dynamic oracle for beam parsing. Only recently developed for
   greedy parsing
\end{points}


\begin{plain}{A non-monotonic EDIT transition}

\begin{figure}
    \centering
    \begin{dependency}[theme=simple]
    \tikzstyle{t}=[text=green,ultra thick,font=\bfseries\itshape]
    \tikzstyle{q}=[text=gray,very thin, dashed]
    \tikzstyle{m}=[font=\bfseries\itshape]
    \tikzstyle{n}=[font=\itshape]
    \begin{deptext}[column sep=.075cm, row sep=.1ex]
        The \& |[q]|red \& |[q]|square \\
    \end{deptext}
    \deproot[edge height=0.7cm, ultra thick]{1}{}
    %\depedge{3}{2}{}
    %\depedge{3}{1}{}
    %\depedge{5}{4}{}
    %\depedge{5}{2}{}
    %\depedge{5}{1}{}
    %\depedge{6}{5}{root}
\end{dependency}
\end{figure}
\end{plain}

\begin{plain}{A non-monotonic EDIT transition}

\begin{figure}
    \centering
    \begin{dependency}[theme=simple]
    \tikzstyle{t}=[text=green,ultra thick,font=\bfseries\itshape]
    \tikzstyle{q}=[text=gray,very thin, dashed]
    \tikzstyle{m}=[font=\bfseries\itshape]
    \tikzstyle{n}=[font=\itshape]
    \begin{deptext}[column sep=.075cm, row sep=.1ex]
        %The \& red \& square \& uh \& rectangle \&  \textsc{r} \\
        The \& red \& |[q]|square \& |[q]|uh \\
        %\& rectangle \&  \textsc{r} \\
    \end{deptext}
    \deproot[edge height=0.7cm, ultra thick]{2}{}
    %\depedge{3}{2}{}
    %\depedge{3}{1}{}
    %\depedge{5}{4}{}
    %\depedge{5}{2}{}
    %\depedge{5}{1}{}
    %\depedge{6}{5}{root}
    \wordgroup{1}{1}{1}{}
\end{dependency}
\end{figure}
\end{plain}



\begin{plain}{A non-monotonic EDIT transition}

\begin{figure}
    \centering
    \begin{dependency}[theme=simple]
    \tikzstyle{t}=[text=green,ultra thick,font=\bfseries\itshape]
    \tikzstyle{q}=[text=gray,very thin, dashed]
    \tikzstyle{m}=[font=\bfseries\itshape]
    \tikzstyle{n}=[font=\itshape]
    \begin{deptext}[column sep=.075cm, row sep=.1ex]
        %The \& red \& square \& uh \& rectangle \&  \textsc{r} \\
        The \& red \& square \& |[q]|uh \& |[q]|rectangle \\
        %\&  \textsc{r} \\
    \end{deptext}
    \deproot[edge height=0.7cm, ultra thick]{3}{}
    %\depedge{3}{2}{}
    %\depedge{3}{1}{}
    %\depedge{5}{4}{}
    %\depedge{5}{2}{}
    %\depedge{5}{1}{}
    %\depedge{6}{5}{root}
    \wordgroup{1}{1}{1}{}
    \wordgroup{1}{2}{2}{}
\end{dependency}
\end{figure}
\end{plain}

\begin{plain}{A non-monotonic EDIT transition}

\begin{figure}
    \centering
    \begin{dependency}[theme=simple]
    \tikzstyle{t}=[text=green,ultra thick,font=\bfseries\itshape]
    \tikzstyle{q}=[text=gray,very thin, dashed]
    \tikzstyle{m}=[font=\bfseries\itshape]
    \tikzstyle{n}=[font=\itshape]
    \begin{deptext}[column sep=.075cm, row sep=.1ex]
        The \& red \& square \& |[q]|uh \& |[q]|rectangle \\
        %\&  \textsc{r} \\
    \end{deptext}
    \deproot[edge height=0.7cm, ultra thick]{3}{}
    \depedge{3}{2}{}
    %\depedge{3}{1}{}
    %\depedge{5}{4}{}
    %\depedge{5}{2}{}
    %\depedge{5}{1}{}
    %\depedge{6}{5}{root}
    \wordgroup{1}{1}{1}{}
\end{dependency}
\end{figure}
\end{plain}


\begin{plain}{A non-monotonic EDIT transition}

\begin{figure}
    \centering
    \begin{dependency}[theme=simple]
    \tikzstyle{t}=[text=green,ultra thick,font=\bfseries\itshape]
    \tikzstyle{q}=[text=gray,very thin, dashed]
    \tikzstyle{m}=[font=\bfseries\itshape]
    \tikzstyle{n}=[font=\itshape]
    \begin{deptext}[column sep=.075cm, row sep=.1ex]
        The \& red \& square \& |[q]|uh \& |[q]|rectangle \\
        %\&  \textsc{r} \\
    \end{deptext}
    \deproot[edge height=0.7cm, ultra thick]{3}{}
    \depedge{3}{2}{}
    \depedge{3}{1}{}
    %\depedge{5}{4}{}
    %\depedge{5}{2}{}
    %\depedge{5}{1}{}
    %\depedge{6}{5}{root}
\end{dependency}
\end{figure}
\end{plain}


\begin{plain}{A non-monotonic EDIT transition}

\begin{figure}
    \centering
    \begin{dependency}[theme=simple]
    \tikzstyle{t}=[text=green,ultra thick,font=\bfseries\itshape]
    \tikzstyle{q}=[text=gray,very thin, dashed]
    \tikzstyle{m}=[font=\bfseries\itshape]
    \tikzstyle{n}=[font=\itshape]
    \begin{deptext}[column sep=.075cm, row sep=.1ex]
        The \& red \& square \& uh \& |[q]|rectangle \&  \textsc{r} \\
    \end{deptext}
    \deproot[edge height=0.7cm, ultra thick]{4}{}
    \depedge{3}{2}{}
    \depedge{3}{1}{}
    %\depedge{5}{4}{}
    %\depedge{5}{2}{}
    %\depedge{5}{1}{}
    %\depedge{6}{5}{root}
    \wordgroup{1}{3}{3}{}
\end{dependency}
\end{figure}
\end{plain}

\begin{plain}{A non-monotonic EDIT transition}

\begin{figure}
    \centering
    \begin{dependency}[theme=simple]
    \tikzstyle{t}=[text=green,ultra thick,font=\bfseries\itshape]
    \tikzstyle{q}=[text=gray,very thin, dashed]
    \tikzstyle{m}=[font=\bfseries\itshape]
    \tikzstyle{n}=[font=\itshape]
    \begin{deptext}[column sep=.075cm, row sep=.1ex]
        The \& red \& square \& uh \& rectangle \&  \textsc{r} \\
    \end{deptext}
    \deproot[edge height=0.7cm, ultra thick]{5}{}
    \depedge{3}{2}{}
    \depedge{3}{1}{}
    %\depedge{5}{4}{}
    %\depedge{5}{2}{}
    %\depedge{5}{1}{}
    %\depedge{6}{5}{root}
    \wordgroup{1}{3}{3}{}
    \wordgroup{1}{4}{4}{}
\end{dependency}
\end{figure}
\end{plain}

\begin{plain}{A non-monotonic EDIT transition}
\begin{figure}
    \centering
    \begin{dependency}[theme=simple]
    \tikzstyle{t}=[text=green,ultra thick,font=\bfseries\itshape]
    \tikzstyle{q}=[text=gray,very thin, dashed]
    \tikzstyle{m}=[font=\bfseries\itshape]
    \tikzstyle{n}=[font=\itshape]
    \begin{deptext}[column sep=.075cm, row sep=.1ex]
        The \& red \& square \& uh \& rectangle \&  \textsc{r} \\
    \end{deptext}
    \deproot[edge height=0.7cm, ultra thick]{5}{}
    \depedge{3}{2}{}
    \depedge{3}{1}{}
    \depedge[edge below]{5}{4}{IP}
    %\depedge{5}{2}{}
    %\depedge{5}{1}{}
    %\depedge{6}{5}{root}
    \wordgroup{1}{3}{3}{}
\end{dependency}
\end{figure}
\end{plain}


\begin{plain}{A non-monotonic EDIT transition}
\begin{figure}
    \centering
    \begin{dependency}[theme=simple]
    \tikzstyle{t}=[text=green,ultra thick,font=\bfseries\itshape]
    \tikzstyle{q}=[text=gray,very thin, dashed]
    \tikzstyle{m}=[font=\bfseries\itshape]
    \tikzstyle{n}=[font=\itshape]
    \begin{deptext}[column sep=.075cm, row sep=.1ex]
        The \& red \& square \& uh \& rectangle \&  \textsc{r} \\
    \end{deptext}
    \deproot[edge height=0.7cm, ultra thick]{5}{}
    \depedge{3}{2}{}
    \depedge{3}{1}{}
    \depedge[edge below]{5}{4}{IP}
    \depedge[green, dashed]{5}{3}{disfl}
    \depedge[red, edge below, dashed]{5}{1}{}
    \depedge[red, edge below, dashed]{5}{2}{}
    %\depedge{6}{5}{root}
    \wordgroup{1}{3}{3}{}
\end{dependency}
\end{figure}
\end{plain}


\begin{plain}{EDIT transition before (above) and after (below)}
\begin{figure}
    \centering
    \begin{dependency}[theme=simple]
    \tikzstyle{t}=[text=green,ultra thick,font=\bfseries\itshape]
    \tikzstyle{q}=[text=gray,very thin, dashed]
    \tikzstyle{m}=[font=\bfseries\itshape]
    \tikzstyle{n}=[font=\itshape]
    \begin{deptext}[column sep=.075cm, row sep=.1ex]
        The \& red \& square \& uh \& rectangle \&  \textsc{r} \\
        The \& red \& \cancel{square} \& uh \& rectangle \&  \textsc{r} \\
    \end{deptext}
    \deproot[edge height=0.7cm, ultra thick]{5}{}
    \depedge{3}{2}{}
    \depedge{3}{1}{}
    \depedge{5}{4}{IP}
    %\depedge[green, dashed]{5}{3}{}
    %\depedge[red, edge below, dashed]{5}{1}{}
    %\depedge[red, edge below, dashed]{5}{2}{}
    %\depedge{6}{5}{root}
    \wordgroup{1}{3}{3}{}
    \wordgroup{2}{1}{1}{}
    \wordgroup{2}{2}{2}{}
    \depedge[edge below]{5}{4}{IP}
\end{dependency}
\end{figure}
EDIT: Mark the top of the stack and its rightward children as disfluent. Restore
leftware children to the stack.
\end{plain}

\begin{points}{More motivating examples for EDIT}
    \p She \emph{had her own} --- \textbf{still had her own}
    \p Well of course \emph{it 's} --- you know, \textbf{it 's}
    \p Fewer than one percent of the \emph{heads of} --- uh, \textbf{chairmen of the board}
    \p 
\end{points}

\begin{plain}{Results}
    \begin{table}
    \begin{tabular}{l|rr}
        \hline
        Model         & UAS  & Disf. F \\
        \hline \hline
        JustParse     & 90.8 & 71.8  \\
        +Feats        & 91.1 & 79.9 \\
        EDIT+Feats    & \textbf{91.4} & \textbf{82.4} \\
        \hline
        Hale et al (2006) & 82.1 & 64.4 \\
        Qian and Liu (2013)         & -- & 84.1 \\
        Johnson and Charniak (2004) & -- & 79.7 \\
        \hline
    \end{tabular}
    \centering
\end{table}
My results are on development set, and not fully comparable. Also I used gold-standard
POS tags.
\end{plain}

\begin{points}{Conclusion and future work}
\p First successful joint parsing and disfluency detection
\p Incremental dependency parsers are well suited to the task
\p Full flexibility in feature functions, due to no dynamic programming
\p Latent edit structures and non-monotonic transition system bring further improvements
\p Incremental model will be able to handle non-gold sentence segmentation
\p Future work: conditional language model for speech recognition
\end{points}

\end{document}
