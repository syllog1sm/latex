% vim: set textwidth=78 fo+=t :

\documentclass[11pt,letterpaper]{article}
\usepackage{acl2013}
\usepackage{amsmath, amsthm}
\usepackage{times}
\usepackage{latexsym}
\usepackage{xspace}
\usepackage{natbib}
\usepackage{tikz-dependency}
\usepackage{placeins}
\usepackage{xcolor}
\usepackage[noend]{algpseudocode}
\usepackage{hyperref}
\usepackage{algorithm}
\usepackage{cancel}
% Alter some LaTeX defaults for better treatment of figures:
    % See p.105 of "TeX Unbound" for suggested values.
    % See pp. 199-200 of Lamport's "LaTeX" book for details.
    %   General parameters, for ALL pages:
    \renewcommand{\topfraction}{0.9}	% max fraction of floats at top
    \renewcommand{\bottomfraction}{0.8}	% max fraction of floats at bottom
    %   Parameters for TEXT pages (not float pages):
    \setcounter{topnumber}{2}
    \setcounter{bottomnumber}{2}
    \setcounter{totalnumber}{4}     % 2 may work better
    \setcounter{dbltopnumber}{2}    % for 2-column pages
    \renewcommand{\dbltopfraction}{0.9}	% fit big float above 2-col. text
    \renewcommand{\textfraction}{0.07}	% allow minimal text w. figs
    %   Parameters for FLOAT pages (not text pages):
    \renewcommand{\floatpagefraction}{0.7}	% require fuller float pages
	% N.B.: floatpagefraction MUST be less than topfraction !!
    \renewcommand{\dblfloatpagefraction}{0.7}	% require fuller float pages

	% remember to use [htp] or [htpb] for placement


\setlength\titlebox{6.5cm}    % Expanding the titlebox


\renewcommand{\tabcolsep}{5pt}

\newcommand{\baseacc}{00.00\xspace}
\newcommand{\sysacc}{00.00\xspace}
\newcommand{\sysimprove}{00.00\xspace}
\newcommand{\las}{\textsc{las}\xspace}
\newcommand{\uas}{\textsc{uas}\xspace}
\newcommand{\pp}{\textsc{pp}\xspace}
\newcommand{\pos}{\textsc{pos}\xspace}
\newcommand{\wsj}{\textsc{wsj}\xspace}
\newcommand{\edittrans}{\textsc{edit}\xspace}

\newcommand{\stacktop}{S$_0$\xspace}
\newcommand{\buffone}{N$_0$\xspace}

\newcommand{\tuple}[1]{$\langle#1\rangle$}
\newcommand{\maybe}[1]{\textcolor{gray}{#1}}
\newcommand{\note}[1]{\textcolor{red}{#1}}
\newcommand{\state}{\mathcal{S}}
\newcommand{\nmae}{\textsc{nmae}\xspace}
\newcommand{\pcfg}{\textsc{pcfg}\xspace}

\newcommand{\szero}{S0\xspace}
\newcommand{\nzero}{N0\xspace}

\newcommand{\szeroh}{S0$_h$\xspace}
\newcommand{\szerohh}{S0$_{h2}$\xspace}
\newcommand{\szeroL}{S0$_L$\xspace}
\newcommand{\szeroLL}{S0$_{L2}$\xspace}
\newcommand{\szeroR}{S0$_R$\xspace}
\newcommand{\szeroRR}{S0$_{R2}$\xspace}
\newcommand{\szeroLzero}{S0$_{L0}$\xspace}
\newcommand{\szeroRzero}{S0$_{R0}$\xspace}

\newcommand{\nzeroL}{N0$_L$\xspace}
\newcommand{\nzeroLL}{N0$_{LL}$\xspace}
\newcommand{\nzeroLzero}{N0$_{L0}$\xspace}

\newcommand{\szeroRedge}{S0$_{re}$\xspace}
\newcommand{\szeroLedge}{S0$_{le}$\xspace}
\newcommand{\nzeroLedge}{N0$_{le}$\xspace}

\newcommand{\sparseval}{\textsc{sparseval}\xspace}

\title{Joint incremental disfluency detection and dependency parsing}

\author{
	Anonymous\\
  	Department\\
  	Institution\\
  	Address\\
  {\tt \small email }\\
}

\date{}

\begin{document}
\maketitle
\begin{abstract}

    We present the first joint model of dependency parsing and
    disfluency detection. A standard incremental beam-search model
    proves well suited to the task, achieving much better parsing and disfluency
    detection scores than previous joint \pcfg models.

We then improve the model further, by introducing features targetting disfluency
detection, and a novel non-monotonic transition system designed for speech repairs.
Our final model achieves TODO\% disfluency $F$-measure; comparable to the current
state-of-the-art and better than previous incremental models.

\end{abstract}

% P1
\section{Introduction}

Most unscripted speech contains filled pauses (ums and uhs) and errors which are
usually edited on-the-fly by the speaker. Disfluency detection is the task of
marking these infelicities in spoken language transcripts. The task has some
immediate value, as disfluencies have been shown to make speech recognition output
much more difficult to read \citep{jones:03}, but has also been motivated as
a module in a natural language understanding pipeline, because disfluencies have
proven problematic for \pcfg parsing models.

One reason for this difficulty could be the effect of disfluencies on the size
of the grammar that the parser has to learn: even though only 5\% of the tokens
in the Switchboard treebank \citep{marcus:93} are disfluent, they increase
the number of production rules in the corpus by over 50\%.\footnote{With
no nodes labelled \textsc{edted}, \textsc{uh} or \textsc{prn}, there are
2,570 unique productions in sections 2 and 3 of Switchboard. With them,
there are 19,015. Punctuation and metadata tokens were removed before both counts.}

%The underlying problem may be that a \pcfg model incorporates assumptions that
%seem a poor fit for learning disfluent sentences, which are examples of linguistic
%performance, rather than linguistic competence.

Instead of a pipeline approach, we adopt a different type of parsing model,
incremental dependency parsing,
which does not rely on a grammar and makes fewer linguistic assumptions.
The only previous work on dependency parsing of speech that we are aware of
is from \citet{jorgensen:07}, who ran the \textsc{malt} parser on the Switchboard
corpus. Since then, dependency parsers have improved in accuracy substantially,
and now rival the best \pcfg models.
%\footnote{The parsers described by
%\citet{zhang:11} and \citet{Charniak05a} both recover unlabelled Stanford basic
%dependencies with about 93.5\% accuracy on the Wall Street Journal.}

Our first experiment is to apply a state-of-the-art incremental dependency parser
`out-of-the-box', with disfluencies simply marked by giving them a distinct
dependency label. A standard beam-search structured perceptron parser
\citep{zhang:cl11,zhang:11} predicted the heads of fluent words
with 89.7\% accuracy, and detected disfluencies with 75.5\% $F$-measure.
We then describe methods of customising the feature set and transition system
of the parser to deal with disfluencies, which together improve these scores
substantially.

Our disfluency features were designed to capture the
`rough copy' structure of some speech repairs, which motivates the
\citet{Johnson04a} noisy channel model.  These features improve disfluency
$F$-measure to 81.1\%.  The ease with which
incremental dependency parsers can incorporate arbitrary features is one of
their principal advantages.

The second improvement is a novel non-monotonic extension to the transition
system. It allows the parser to create temporary edges, in order to deal with
speech repairs where the head of an apparent constituent is disfluent, but some
of its apparent leftward children are not.

The joint system achieves competitive accuracy at both disfluency detection and
parsing. It also has other important advantages:

\begin{itemize}
    \item Because the parser is incremental, it does not require sentence segmentation
        as a pre-process. Segmenting spoken utterances is a difficult unsolved
        problem, that will probably benefit from syntactic information. Our parser
        is the first to make this possible.

    \item The system is also quite efficient, processing 488 tokens per second.
        Little work has been done on making disfluency detection systems efficient,
        which has greatly limited their use. Previous models have required either
        large ngram language models \citep{zwarts:11}, a \pcfg syntactic language
        model \citep{Johnson04a}, or several passes over the data \citep{qian:13}.

    \item Finally, by casting the problem of disfluency detection as a minimally
           different dependency parsing task, our system will be able to exploit
           future improvements to incremental dependency parsing —-- an area of
           research that is currently very active.
\end{itemize}

\section{Switchboard disfluency annotations}

The Switchboard portion of the Penn Treebank \citep{marcus:93} consists of
telephone conversations between strangers about
an assigned topic.  Two annotation layers are provided: one for syntactic
bracketing (\textsc{mrg} files),
and one for disfluencies (\textsc{dps} files).  The disfluency layer marks
elements with little or no syntactic function, such as filled pauses and discourse
markers, and annotates speech repairs using the \citet{shriberg:94} system of
reparandum/interregnum/repair.
The main way the \textsc{mrg} files mark disfluencies is with a node labelled
\textsc{edited}.  The idea behind this annotation is to mark text which, if
excised, will lead to a grammatical sentence.

%The \textsc{dps} reparandum/interregnum/repair boundaries are represented in the
%\textsc{mrg} files using meta-data tokens.  However, there is an additional
%complication: the two layers are segmented differently, so if the treebank
%sentences are assumed to be independent of each other, the
%reparandum/interregnum/repair annotations in them do not always make sense.

The two layers have high but imperfect agreement over
what tokens they mark disfluent: the \textsc{dps} files call
32,310 tokens speech repairs, while 32,742 are marked disfluent in the treebank,
with 33,720 tokens marked disfluent in at least one layer.

\subsection{Dependency conversion}

As is standard in recent statistical dependency parsing work, we acquire our
gold-standard dependencies by converting phrase-structure trees.
We used the 2013-04-05 version of the Stanford dependency converter \citep{stanford_deps}.
At first we feared that the filled pauses, disfluencies and meta-data tokens in
the Switchboard corpus might disrupt the conversion process, by making it more
difficult for the converter to recognise the underlying production rules.

To test this, we prepared two versions of the corpus, both with oracle pre-processing:
one where \textsc{edited} nodes, filled pauses and meta-data were cleaned before
the trees were sent through the Stanford converter, and one where the disfluency
cleaning was performed after the dependency conversion. The resulting corpora
were largely identical: 99.54\% of unlabelled and 98.7\% of labelled dependencies
were the same.

The fact that the Stanford converter is robust to the disfluencies is good news
for our baseline model. In our \textsc{edit} model, the structure of disfluent
constituents is under-specified; but for the baseline model, we use the structure
output by the Stanford converter.

We follow previous work on disfluency detection in lower-casing the text and
removing punctuation and partial words\textsc{Words tagged XX and words ending in
`'.}.  We also filter out one- and two-token sentences.

We found that two additional simple pre-processes improved our results:

\begin{enumerate}
    \item Discard all `um' and `uh' tokens;
    \item Merge `you know' and `i mean' into single tokens.
\end{enumerate}

These pre-processes can be completed on the input string without losing information:
none of the `um' or `uh' tokens are semantically significant, and
the bigrams \emph{you know} and \emph{you mean} have a dependency between the two
tokens over 99.9\% of the times they occur in the treebank. We found that `um'
and `uh' were not useful clues about when speakers might become disfluent, and
we doubted that there was any utility in predicting a syntactic head for them.
More nuanced handling of speech-specific multi-word expressions seems an interesting
topic for future work.

\section{Transition-based dependency parsing}

A transition-based parser predicts the syntactic structure of a sentence incrementally,
by making a sequence of classification decisions.  We follow the architecture of
\citet{zhang:cl11}, who use beam-search for decoding, and the averaged
perceptron algorithm for learning.  Despite its simplicity, this type of parser
has produced highly competitive results on the Wall Street Journal: with the
extended feature set described by \citet{zhang:11}, the parser achieves 93.5\%
unlabelled attachment score on Stanford basic dependencies.  The \citet{Charniak05a}
reranking parser produces similar accuracy when its constituency trees are post-processed
by the Stanford dependency converter.

Briefly, the transition-based parser consists of a state (sometimes termed a
configuration) which is sequentially manipulated by actions chosen from a set
termed the transition system. We use the arc-eager transition system
\citep{nivre:03,nivre:cl}.

We use a notation in which the stack items are indicated by S$i$, with S0 being
the top of the stack, S1 the item previous to it and so on. Similarly, buffer
items are indicated as N$i$, with N0 being the first item on the buffer.
In the initial configuration the stack is empty, and the buffer contains the words
of the sentence followed by an artificial \textsc{root} token, as suggested by
\citet{nivre:squib}.
In the final configuration the buffer is empty and the stack contains the 
\textsc{root} token.
%The set of arcs in the final configuration is the parse tree.

There are four parsing actions (\textbf{S}hift, \textbf{L}eft-Arc,
\textbf{R}ight-Arc and Re\textbf{d}uce, abbreviated as S,L,R,D respectively)
that manipulate stack and buffer items. The Shift action pops the first item
from the buffer and pushes it on the stack (the Shift action has a natural
precondition that the buffer is not empty, as well as a precondition that
\textsc{root} can only be pushed to an empty stack). The Right-Arc action is
similar to the Shift action, but it also adds a dependency arc (S0, N0),
with the current top of the stack as the head of the newly pushed item (the Right
action has an additional precondition that the stack is not empty). The Left-Arc
action adds a dependency arc (N0, S0) with the first item in the buffer as the
head of the top of the stack, and pops the stack (with a precondition that the
stack and buffer are not empty, and that S0 is not assigned a head yet). Finally,
the Reduce action pops the stack, with a precondition that the stack is not empty
and that S0 is already assigned a head.

\section{Features for the joint parser}

Our baseline parser uses the feature set described by \citet{zhang:11}.
To describe the features, we use the following notation. First, we name the
12 context tokens shown in Figure ??. These are the word on top of the stack, its
left and right subtree, its parent and its grandparent; the first word of the buffer
and its leftward subtree (it cannot have a rightward subtree); and the next two
tokens of the buffer (these cannot have subtrees or parents).

Atomic features consist of the word, part-of- speech tag, or dependency label
for these tokens, and feature templates often consist of multiple fea- ture atoms
in combination. We refer the reader to \citet{zhang:11} for the specific feature
templates used.

\subsection{Brown cluster features}

The Brown clustering algorithm \citep{clustering} provides a well known source
of semi-supervised features. The clustering algorithm is run over a large sample
of unlabelled data, to generate a type-to-cluster map. This mapping is then used
to generate features that sometimes generalise better than lexical features
\citet{turian}.

\citet{koo:10} found that Brown cluster features greatly improved the performance
of a graph-based dependency parser. On our transition-based parser, Brown cluster
features bring a small but statistically significant improvement on the \textsc{wsj}
task (0.1-0.3\% \textsc{uas}).  Other developers of transition-based parsers
seem to have found similar results (personal communication).

Since the Brown cluster type-mapping computed by \citet{turian:10} is easily
available\footnote{\url{www.metaoptimize.com}}, the features are very simple
to implement and cheap to compute, so we see little reason not to include them
in the parser.

Our templates follow \citet{koo:10} in including features for full clusters as
well as cluster prefixes. We adapt their templates to transition-based parsing
by replacing `head' with `item on top of the stack' and `child' with `first word
of the buffer'. The exact templates can be found in the supplementary material.

\subsection{Rough copy features}

\citet{Johnson04a} point out that in speech repairs, the repair is often a rough
copy of the reparandum.  The simplest case of this is where the repair is a single
word repetition. In more complex cases, some of the wording or grammatical
structure will be the same, and the rest will be different.

To capture this regularity, we first extend the feature-set to track three new
\emph{context tokens}:
\begin{enumerate}
    \item \szeroRedge : The rightmost edge of \szero 's yield;
    \item \szeroLedge : The leftmost edge of \szero 's yield;
    \item \nzeroLedge : The leftmost edge of \nzero 's yield.
\end{enumerate}

The \emph{yield} of a word is the span of words headed by it.  If a word has
no leftward children, it will be its own left-edge, and similarly it will be
its own rightward edge if it has no rightward children. Note that the token
\szeroRedge is necessarily immediately before \nzeroLedge, unless some of the
tokens between them are disfluent.

The yield-edge tokens can be updated during parsing for the word on top of the
stack and the first word of the buffer.
It is difficult to ensure that other words
have their yield-edges marked correctly, but that is not important
for our features.

The Left-Arc move makes \szero the leftmost child of \nzero. The left yield-edge of
\nzero is therefore the left yield-edge of \szero. The Right-Arc move makes \nzero
the rightmost child of \szero. \nzero cannot have a rightward subtree, so it must
be the rightmost child of \szero.  After a Right-Arc, \nzero will also become
new rightmost edge of all ancestors of \szero.  When the Reduce move is used,
we set the right yield-edge of S1 to be the right yield-edge of \szero.

We calculate a simple form of rough-copy feature using the \szeroLedge and
\nzeroLedge tokens.  The features ask:

\begin{itemize}
    \item How long is the \emph{prefix word match} between \szeroLedge...\szero
          and \nzeroLedge...\nzero?
    \item How long is the \emph{prefix POS tag match} between \szeroLedge...\szero
          and \nzeroLedge...\nzero?
    \item Do the words in \szeroLedge...\szero and \nzeroLedge...\nzero match
          exactly?
    \item Do the POS tags in \szeroLedge...\szero and \nzeroLedge...\nzero match
          exactly?
\end{itemize}

For the prefix-length features, we adopt the simple discretisation strategy of
introducing a different binary-valued feature for each value up to 5, and another
bin for 5+.\footnote{We also tried the more nuanced strategy of having a bin for each
\emph{range} up to 5, so that the bins ask e.g. `Is the prefix \emph{at
least} two words long?', instead of `Is the prefix \emph{exactly} two words long?'.
This turned out to make no difference, so we used the simpler scheme.}

\subsection{Match features}

This class of features ask which pairs of the \emph{context tokens} match, in
word or \pos tag.  The context tokens in the \citet{zhang:11} feature set
are the top of the stack (\szero), its head and grandparent (\szeroh, \szerohh),
its two left- and rightmost children (\szeroL, \szeroLL, \szeroR, \szeroRR), the
first three words of the buffer (\nzero, N1, N2), and the two leftmost children
of \nzero (\nzeroL, \nzeroLL).
We extend this set with the \szeroLedge, \szeroRedge and \nzeroLedge tokens
described above, and also the first left and right child of \szero and \nzero
(\szeroLzero, \szeroRzero, \nzeroLzero).

All up, there are 18 context tokens, so ${18 \choose 2}~=~153$ token pairs.
For each pair of these tokens, we add two binary features, indicating whether the
two tokens match in word form or \pos tag.  We also have two further classes of
features: if the words do match, a feature is added indicating the word form;
if the tags match, a feature is added indicating the tag. These finer grained
versions help the model adjust for the fact that some words can be duplicated
in grammatical sentences (e.g. `that that'), while most rare words cannot.

\subsection{Edited neighbour features}

Disfluencies are usually
string contiguous, even if they do not form a single constituent.  We help the
parser exploit this by adding features asking whether the word or pair of words
immediately before \nzero were marked disfluent. A similar pair of features is
added for the word or pair of words immediately after \szero.

\section{A non-monotonic Edit transition}

One of the reasons that joint disfluency detection and parsing is hard is due to
what ? terms helical dependency structures. Figure ?? shows a simple example.
The repair square replaces the reparan- dum5 rectangle, and must be attached to
the rest of the fluent sentence.

The problematic dependencies, shown in dotted lines, are between `rectangle'
and `a' and `red'. These arcs are not exactly correct, as we do not want them in
a final projective tree, but they are also not exactly incorrect, in the way that
an arc from `a' to `pass' or `me' would be.

For a transition-based model, learning to avoid these arcs is very difficult.
Unless the interruption point is within the look-ahead into the buffer, the
incremental model cannot guess that ‘rect- angle’ will later prove disfluent,
so trying to learn disfluency-specific behaviour at this point seems a bad strategy.
What we want instead is for the parser to apply whatever it has learned about English
grammar to this decision, and later arrive at a state that is inconsistent with
that grammar, forcing it to mark rectangle as disfluent.

We achieve this by adding a new transition, \edittrans, to the arc-eager system.
The \edittrans transition marks the top word of the stack and its rightward subtree
as disfluent, but returns its leftward children to the stack, and deletes a
dependency to it if one exists.

Figure ?? shows a before-and-after example. In the configuration above, rectangle
is on top of the stack (indicated by circling the word). After the \textsc{edit}
transition, it is marked as disfluent (crossed), the arc to it is deleted, and its
leftward children (empha and red) are restored to the stack. The substructures
of these children are not rebuilt —-- we do not delete the arc from bright to red,
or restore bright to the stack.

Why revisit the leftward children, but not the right? The logic here is that we are
concerned about dependencies which might be ‘mirrored’ be- tween the reparandum
and the repair. The rightward subtree of the disfluency might well be incorrect,
but if it is, it would still be incorrect if the word on top of the stack weren’t
disfluent —-- so we regard these as parsing errors that we will train our model
to avoid. In contrast, avoiding the left- ward arcs would require the parser to
predict that the head is disfluent, when it has not necessarily seen any evidence
indicating that.

The EDIT transition makes the transition system non-monotonic, in the terms of
\citet{goldberg:12},
who showed that a greedy arc-eager parser became more accurate when given a limited
ability to correct for incorrect transitions. We use a beam-search parser, which
is well equipped to correct for the cases which the non-monoticity they introduce
helps with. We therefore restrict our parser’s non-monotonic behaviour to the
EDIT transition.

The \edittrans transition introduces substantial spurious ambiguity: there are multiple
transition sequences that could yield the gold-standard dependencies. We therefore
adopt the `dynamic oracle' training method of \citet{goldberg:12} described in
Section 3.1.


\subsection{Motivating example}

\begin{figure*}
    \small
    \centering
    \begin{tabular}{llc}
    % 1 R
        1 & $\langle R \rangle $ & \begin{dependency}[theme=simple, segmented edge, edge unit distance=1.0ex]
        \begin{deptext}[column sep=.075cm, row sep=.1ex]
            I \& noticed \& that \& $\|$ the \& \emph{monthly} \& \emph{salary} \& starting \& average \& monthly \& \emph{salary} \& salary \& for \& engineers \hfill \\
        \end{deptext}
        \wordgroup{1}{2}{2}{}
        \wordgroup{1}{3}{3}{}
        \depedge{2}{1}{}
        \depedge{2}{3}{}
    \end{dependency}\\[-2.0ex]
    % 2 SS
    2 & $\langle DSS \rangle$ & \begin{dependency}[theme=simple, segmented edge, edge unit distance=1.0ex]
        \begin{deptext}[column sep=.075cm, row sep=.1ex]

            I \& noticed \& that \& the \& \emph{monthly} \& $\|$ \emph{salary} \& starting \& average \& monthly \& \emph{salary} \& salary \& for \& engineers \hfill \\
        \end{deptext}
        \wordgroup{1}{2}{2}{}
        \wordgroup{1}{4}{4}{}
        \wordgroup{1}{5}{5}{}
        \depedge{2}{1}{}
        \depedge{2}{3}{}
    \end{dependency}\\[-2.0ex]
 
    % 3 LL
    3 & $\langle LL \rangle$ & \begin{dependency}[theme=simple, segmented edge, edge unit distance=1.0ex]
        \begin{deptext}[column sep=.075cm, row sep=.1ex]
            I \& noticed \& that \& the \& \emph{monthly} \& $\|$ \emph{salary} \& starting \& average \& monthly \& \emph{salary} \& salary \& for \& engineers \\
        \end{deptext}
        \wordgroup{1}{2}{2}{}
        %\wordgroup{1}{4}{4}{}
        %\wordgroup{1}{5}{5}{}
        \depedge{2}{1}{}
        \depedge{2}{3}{}
        \depedge{6}{5}{}
        \depedge{6}{4}{}
    \end{dependency}\\[-2.0ex]
 
    % 4 R
4 & $\langle R \rangle$ & \begin{dependency}[theme=simple, segmented edge, edge unit distance=1.0ex]
        \begin{deptext}[column sep=.075cm, row sep=.1ex]
            I \& noticed \& that \& the \& \emph{monthly} \& \emph{salary} \& $\|$ starting \& average \& monthly \& \emph{salary} \& salary \& for \& engineers \\
        \end{deptext}
        \wordgroup{1}{2}{2}{}
        %\wordgroup{1}{4}{4}{}
        %\wordgroup{1}{5}{5}{}
        \depedge{2}{1}{}
        \depedge{2}{3}{}
        \depedge{6}{5}{}
        \depedge[edge unit distance=0.9ex]{6}{4}{}

        \depedge[edge unit distance=0.3ex, edge below]{2}{6}{}
        \wordgroup{1}{6}{6}{}
    \end{dependency}\\[-2.0ex]
 
    % 5 SSSLLL
5 & $\langle SSSLLL \rangle$ & \begin{dependency}[theme=simple, segmented edge, edge unit distance=1.0ex]
        \begin{deptext}[column sep=.075cm, row sep=.1ex]
            I \& noticed \& that \& the \& \emph{monthly} \& \emph{salary} \& starting \& average \& monthly \& $\|$ \emph{salary} \& salary \& for \& engineers \\
        \end{deptext}
        \wordgroup{1}{2}{2}{}
        %\wordgroup{1}{4}{4}{}
        %\wordgroup{1}{5}{5}{}
        \depedge{2}{1}{}
        \depedge{2}{3}{}
        \depedge{6}{5}{}
        \depedge[edge unit distance=0.9ex]{6}{4}{}

        \depedge[edge unit distance=0.3ex, edge below]{2}{6}{}
        \wordgroup{1}{6}{6}{}

        \depedge[edge unit distance=0.9ex]{10}{9}{}
        \depedge[edge unit distance=0.8ex]{10}{8}{}
        \depedge[edge below, edge unit distance=0.3ex]{10}{7}{}
    \end{dependency}\\[-2.0ex]
    % 6 E
    6 & $\langle \textbf{E} \rangle$ & \begin{dependency}[theme=simple, segmented edge, edge unit distance=1.0ex]
        \begin{deptext}[column sep=.075cm, row sep=.1ex]
            I \& noticed \& that \& the \& \emph{monthly} \& \cancel{\emph{salary}} \& starting \& average \& monthly \& $\|$ \emph{salary} \& salary \& for \& engineers \\
        \end{deptext}
        \wordgroup{1}{2}{2}{}
        \wordgroup{1}{4}{4}{}
        \wordgroup{1}{5}{5}{}
        \depedge{2}{1}{}
        \depedge{2}{3}{}

        \depedge[edge unit distance=0.9ex]{10}{9}{}
        \depedge[edge unit distance=0.8ex]{10}{8}{}
        \depedge[edge below, edge unit distance=0.3ex]{10}{7}{}
    \end{dependency}\\[-2.0ex]
 
    % 7 E
 7 & $\langle \textbf{E} \rangle$ & \begin{dependency}[theme=simple, segmented edge, edge unit distance=1.0ex]
        \begin{deptext}[column sep=.075cm, row sep=.1ex]
            I \& noticed \& that \& the \& \cancel{\emph{monthly}} \& \cancel{\emph{salary}} \& starting \& average \& monthly \& $\|$ \emph{salary} \& salary \& for \& engineers \\
        \end{deptext}
        \wordgroup{1}{2}{2}{}
        \wordgroup{1}{4}{4}{}
        \depedge{2}{1}{}
        \depedge{2}{3}{}

        \depedge[edge unit distance=0.9ex]{10}{9}{}
        \depedge[edge unit distance=0.8ex]{10}{8}{}
        \depedge[edge below, edge unit distance=0.3ex]{10}{7}{}
    \end{dependency}\\[-2.0ex]
 
    % 8 LS
 8 & $\langle LS \rangle$ & \begin{dependency}[theme=simple, segmented edge, edge unit distance=1.0ex]
        \begin{deptext}[column sep=.075cm, row sep=.1ex]
            I \& noticed \& that \& the \& \cancel{\emph{monthly}} \& \cancel{\emph{salary}} \& starting \& average \& monthly \& \emph{salary} \& $\|$ salary \& for \& engineers \\
        \end{deptext}
        \wordgroup{1}{2}{2}{}
        %\wordgroup{1}{4}{4}{}
        \depedge{2}{1}{}
        \depedge{2}{3}{}

        \depedge[edge unit distance=0.9ex]{10}{9}{}
        \depedge[edge unit distance=0.8ex]{10}{8}{}
        \depedge[edge below, edge unit distance=0.3ex]{10}{7}{}
        \depedge[edge below, edge unit distance=0.3ex]{10}{4}{}
        \wordgroup{1}{10}{10}{}
    \end{dependency}\\[-2.0ex]
 
    % 9 E
    9 & $\langle \textbf{E} \rangle$ & \begin{dependency}[theme=simple, segmented edge, edge unit distance=1.0ex]
        \begin{deptext}[column sep=.075cm, row sep=.1ex]
            I \& noticed \& that \& the \& \cancel{\emph{monthly}} \& \cancel{\emph{salary}} \& starting \& average \& monthly \& \cancel{\emph{salary}} \& $\|$ salary \& for \& engineers \\
        \end{deptext}
        \wordgroup{1}{2}{2}{}
        \wordgroup{1}{4}{4}{}
        \depedge{2}{1}{}
        \depedge{2}{3}{}

        \wordgroup{1}{7}{7}{}
        \wordgroup{1}{8}{8}{}
        \wordgroup{1}{9}{9}{}
        %\depedge[edge unit distance=0.9ex]{10}{9}{}
        %\depedge[edge unit distance=0.8ex]{10}{8}{}
        %\depedge[edge below, edge unit distance=0.3ex]{10}{7}{}
        %\depedge[edge below, edge unit distance=0.3ex]{10}{4}{}
        %\wordgroup{1}{10}{10}{}
    \end{dependency}\\[-2.0ex]
     10 &    & \begin{dependency}[theme=simple, segmented edge, edge unit distance=1.0ex]
        \begin{deptext}[column sep=.075cm, row sep=.1ex]
            I \& noticed \& that \& the \& \cancel{\emph{monthly}} \& \cancel{\emph{salary}} \& starting \& average \& monthly \& \cancel{\emph{salary}} \& $\|$ salary \& for \& engineers \\
        \end{deptext}
        \wordgroup{1}{2}{2}{}
        \depedge{2}{1}{}
        \depedge{2}{3}{}

        \depedge[edge unit distance=0.9ex]{11}{9}{}
        \depedge[edge unit distance=0.8ex]{11}{8}{}
        \depedge[edge below, edge unit distance=0.3ex]{11}{7}{}
        \depedge[edge below, edge unit distance=0.3ex]{11}{4}{}
        \depedge[edge unit distance=0.3ex]{2}{11}{}
        \depedge{11}{12}{}
        \depedge{12}{13}{}
        %\wordgroup{1}{10}{10}{}
    \end{dependency}\\[-2.0ex]
 
    \end{tabular}
    \caption{\small Example from the development data where the Edit transition improves accuracy. Words on the stack are shown circled, while words marked disfluent with the Edit transition are struck-through. The start of the buffer is marked with $\|$}.
\end{figure*}

Figure 2 shows part of a sentence from the development data that contains several
difficult disfluencies, of the type the Edit transition is designed to address.
Each numbered line shows a state along the final path predicted by the parser.
The actions performed to transform the previous state into the current one are
listed in angle brackets. Words on the stack are circled, and dependencies are
shown as arcs from the head to the child. The $\|$ symbol shows where the buffer
begins.

Line 1 shows a state resulting resulting from a Right-Arc (R) from noticed to that.
Previously, the parser had performed a Left-Arc from noticed to I, which is no
longer on the stack. The itali- cised words in the buffer are disfluent, and will
be deleted using the \edittrans transition. Deletions will be marked by strike-through.

Line 2 shows the state after the Shift move (S)
move has pushed the and monthly onto the stack. It would be correct to apply the
Edit transition to this configuration, but that is only one of several moves that
the dynamic oracle (Sections ??) would assign zero-cost during training.

The model selects a different zero-cost action, Left-Arc, which pops monthly and
attaches it to salary. The parser then performs another Left- Arc, attaching the
fluent word the to the disfluent word salary, resulting in the state at Line 3.

The critical difference between the EDIT model and the baseline is that this type
of arc is also zero- cost in the EDIT model, because we will later put the back
onto the stack. Our argument is that from the configuration in Line 2, the Left-Arcs
are the natural choice, and a loss-function that penalises this move will be
unlearnable. The state resulting from these two actions is shown in Line 3.

The parser’s next action is a Right-Arc from the fluent noticed to the disfluent
\emph{salary}, pushing salary onto the stack. This action is also zero-cost,
even though it creates an erroneous arc, because we’ll later have the chance to
delete the arc via the EDIT transition. The parser then performs a series of
Shift and Left-Arc moves, attaching the fluent tokens starting, monthly and
average to the sec- ond disfluent salary token. These arcs are all also zero-cost,
and are further examples of the type of arc that motivates the \edittrans transition.

The parser is now (Line 4) in a state from which it can confidently predict that
the salary token on top of the stack is disfluent, as there is a matching word at
the start of the buffer that isn’t one of the small set of words that are often
repeated in the training data. The word-match features will also find a match
between the tokens S0L0 and N0L0 (first left child of S0 and first left child
of N0).

Line 5 shows the state that results from the \edittrans transition. It marks salary as
disfluent, deletes all arcs to and from it, pops it from the stack, and restores
its leftward children to the stack — in this case, monthly. The parser’s next
action is to mark monthly as disfluent. A Left-Arc, Shift, or even Right-Arc
would all have been zero-cost from this state, because the word on top of the
stack and the word at the start of the buffer are both disfluent. Instead, the
word-match between S0 and N0L0, combined with the fact that the token after S0
was marked disfluent, leads the parser to mark monthly as also disfluent.

The parser next applies the Left-Arc transition from the disfluent salary to the
fluent the, again at zero-cost, even though the arc is incorrect. The Shift move
then puts the parser in the state shown on Line 7: the disfluent salary is at
the top of the stack, with several fluent words as leftward chil- dren, and its
repair is at the start of the buffer. The word-match feature between S0 and N0
fires, encouraging the parser to select the EDIT transition, which restores the,
starting, average and monthly to the stack, resulting in the state on Line 9.

Having successfully marked the disfluencies, the parser performs a series of
Left-Arcs, before correctly attaching salary to noticed via the Right- Arc,
and happens to correctly attach the preposi- tional phrase as well.

\subsection{Dynamic oracle training}

An essential component when training a transition-based parser is an oracle which,
given a gold-standard tree, dictates the sequence of moves a parser should make
in order to derive it. Traditionally, these oracles are defined as functions from
trees to sequences, mapping a gold tree to a single sequence of actions deriving it,
even if more than one sequence of actions derives the gold tree. During training,
the model suffers a loss if its prediction departs from this single 
sequence, even if it lies on an alternate path to the gold-standard tree.

The oracle that \citet{goldberg:12} define, which they term a \emph{dynamic} oracle,
instead calculates how many gold-standard arcs will be newly unreachable after a
given action has been performed. This allows the model to be trained with a
loss function that refers to the dependency structure, instead of the transition
history.  

Another way to look at this is to say that the derivation structure is left latent,
instead of being learnt from direct supervision.  
We therefore adapt the learning
algorithm of \citet{zettlemoyer:07}, in order to use the dynamic oracle with
a global model, which has not yet been done.

Briefly, \citeauthor{zettlemoyer:07} train a global perceptron model to predict
lambda calculus representations from sentence strings, with Combinatory
Categorial Grammar derivations as a latent structure.  Ignoring the parts of
their algorithm which deal with the lexicon, and generalising a little, the algorithm is:

\begin{enumerate}
\item Search for the model's best hypothesis given the input.
\item If the hypothesis satisfies the supervision constraints, move on to the
      next example.
\item Otherwise, search again, this time constraining the search to ensure the
    candidate \emph{will} satisfy the supervision criteria.
\item Perform a weight update for the model hypothesis and the gold-standard
      candidate.
\end{enumerate}

For us, the supervision constraint is that each action must be zero-cost.  So,
we first run our beam-search decoder, just as at parse time (Step 1),
except that we consult the gold-standard dependencies and disfluency information
to compute a cost for each candidate in the beam, via the dynamic oracle (for Step 2).
If the top-ranked derivation is zero-cost, we perform no update (Step 2).
Otherwise, we search again, but
ensure that no actions with non-zero cost are introduced into the beam (Step 3).
Our weight update is the standard one, from \citet{collins:02}: count how often
each feature/class pair occurs in the gold-standard and predicted sequences,
and update the weights by their difference.

Note that neither search process is exact, and in fact we may find a gold-standard
candidate that scores higher than the predicted candidate, if the gold candidate
were pruned during search.  If handled naively, this would trigger what \citet{huang:12}
refer to as an `invalid update'.  The early-update strategy of \citet{collins:02}
is one way to avoid this, but we adopt the maximum violation update strategy
described by \citet{huang:12} instead, as it requires fewer training iterations.

\subsection{Path length normalisation, worst-case complexity}

One problem introduced by the \textsc{edit} transition is that the number of
actionss applied to a sentence is no longer constant --- it is no longer guaranteed
to be $2n-1$, for a sentence of length $n$. When the \textsc{edit} transition is
applied to a word with leftward children, those children are returned to the stack,
and processed again.

This is a problem for the perceptron model, because the model tends to assign
high scores to its top prediction.\footnote{Recall that each training instance
involves adding 1.0 to the weights for the correct class, and -1.0 to the weights
for the predicted class, over several iterations. The correct class stays the
same, while the predictions may vary, so the negative weight tends to be split
among multiple classes. This is why the top prediction on an instance almost
always has a high positive weight.}
We thus observed a problem when comparing paths of different lengths, at the end
of the sentence. Candidate that proposed \textsc{edit} transitions were longer,
so the sum of their scores tended to be higher.

The same problem has been observed during incremental \textsc{pcfg} parsing,
by \citet{zhang:13}.  They introduce an additional transition, \textsc{idle},
to ensure that paths are the same length. So long as one candidate in the beam
is still being processed, all other candidates apply the \textsc{idle} transition.

We adopt a simpler solution: we rank candidates in the beam by their mean
transition score, instead of the sum of their scores. Our logic is that the
problem is in the figure-of-merit used during decoding, not necessarily with the
weights assigned by the model.

Interestingly, \citet{zhang:13} report that they tried exactly this, and that it
was less effective than their solution. We found the opposite: the features
associated with the \textsc{idle} transition were uninformative (the state is at
termination, so the stack and buffer are empty), and had nothing to do with how
many edit transitions were earlier applied.

This is also a good place to note the effect of the \textsc{edit} transition
on the algorithm's worst-case time complexity. TODO: calculation. We note that
the issue is not of practical importance; our empirical parse times remain linear.

\section{Experiments}

\subsection{Evaluating parse accuracy on disfluent text}

Our parser evaluation is modelled on the \sparseval metric \citep{sparseval}.
However, we wanted to use the Stanford dependency scheme, to keep the system
close to standard practice for Wall Street Journal dependency parsing, so we
do not use the \sparseval tool itself.  Our evaluation script is distributed
in the supplementary materials.

We follow \sparseval in not excluding arcs from disfluent words in the parser
evaluation.  Links from fluent words incorrectly marked disfluent are marked as
incorrect.

\subsection{Parsing with oracle disfluencies}
\label{sec:oracle}
This experiment investigates how much disfluencies disrupt parsing accuracy,
by training and evaluating a parser on a version of Switchboard with oracle
disfluency detection.  Every word labelled as a disfluency\footnote{That is, those
part of the yield of a node labelled \textsc{edited} in the \textsc{ptb} annotation.}
is removed, prior to training and evaluation.

\subsection{Parsing with a disfluency pipeline}
\label{sec:pipeline}
The accuracy of incremental dependency parsers is well established on the Wall
Street Journal, but there are no dependency parsing results in the literature
that make it easy to put our joint model's parsing accuracy into context.

We therefore compare our joint model to a pipeline approach, using a strong 
syntax-based disfluency detection system.  We used the \citet{Johnson04a}
noisy-channel model, with the \citet{Charniak01a} parser as a syntactic language
model.  We chose this system because it is an extension of the \citet{Charniak01b}
boosting classifier, which is the only disfluency detection system studied as
a pre-processor to a syntactic parser.  The accuracy of the model has improved
since \citet{Johnson04a}, as it now includes the additional features described by
\citet{zwarts:11}.  In fact, its accuracy is higher than the previous state-of-the-art.

The parser is first trained on the cleaned corpus, as described in section
\ref{sec:oracle}. The \citet{Charniak01a} is cross-fold trained on the Switchboard
corpus, and then scores candidates produced by the \citet{Johnson04a} noisy-channel
model.  A MaxEnt reranker then considers the channel model and language model scores,
in addition to features described by \citet{zwarts:11}. We then parse the string
that results from omitting words marked as disfluent by top-ranked hypothesis.

As complicated as this pipeline is, and as high-scoring as the disfluency detector
is, the configuration is arguably still slightly unfair, even though it matches
the architecture used by \citet{Charniak01a}. The parser has been
trained on text with oracle pre-processing, but at run-time it is exposed to
text with some disfluencies. The model would probably be more accurate if we
introduced \emph{another} cross-fold training step, so that the parser is trained
on text with some disfluencies.

We consider that a subject for future work, because the natural extension of
that approach is to adopt a form of `pipeline iteration'
\citet{hollingshead:07}. One could just as easily use \emph{our} model as the
disfluency detection pre-process, and chain together several iterations to
successively clean the text.

\subsection{Data set}

We use the Switchboard portion of the Penn Treebank \citep{marcus:93} for our
experiments, which consists of telephone conversations between strangers about
an assigned topic.  Two annotation layers are provided: one for syntactic bracketing,
and one for disfluencies.  The disfluency layer marks elements with
little or no syntactic function, such as filled pauses and discourse markers,
and annotates speech repairs using the \citet{shriberg:94} system of
reparandum/interregnum/repair.

The convention within the syntactic brackets is to cover speech-repairs with
the node \textsc{edited}.  The complication is that the two layers were annotated
separately, and do not always overlap, as the project was halted before they
could be fully reconciled. This complicates comparison with previous work, as
researchers interested in parsing have tended to use only the \textsc{edited}
nodes as disfluency markers \citep{Charniak01a,jorgensen:07}, while those solely
interested in disfluency detection have tended to use the richer annotations in
the \textsc{dps} files.

We found that following the \textsc{dps} file annotations led to bad
gold-standard parses when the annotation disagreed with the \textsc{edited}
nodes.  In the training data, TODO words are under \textsc{edited} nodes, and
TODO of them are also marked as disfluent in the \textsc{dps} annotations
imported into the \textsc{mrg} files. There are TODO words that the \textsc{dps}
annotations mark as disfluent that are \emph{not} under \textsc{edited} nodes.

\begin{table}
    \centering
    \small
    \begin{tabular}{l|rrr|rr}
        & P & R & F & \textsc{uas} & \textsc{las} \\
        \hline \hline
        Parser      & 81.9 & 70.1 & 75.5 & 89.7  & 86.8 \\
        +Clusters   & 81.5 & 70.7 & 75.7 & 89.8  & 86.8 \\
        +Features   & 86.6 & 76.2 & 81.1 & 90.1  & 87.1 \\
        +Transition & 91.9 & 80.0 & 85.6 & 90.6  & 87.5 \\
        \hline
        Pipe (Sec. \ref{sec:pipeline})  & 89.4 & 80.3 & 84.6 & 90.4 & 87.3 \\
        \hline       
        Oracle (Sec. \ref{sec:oracle})  & 100 & 100 & 100 & 91.7 & 88.5 \\
        \hline
    \end{tabular}
\caption{Development accuracies for disfluency detection and parse accuracy.}
\end{table}



\section{Results}

\begin{table}
\centering
\begin{tabular}{l|r}
    System          & \textsc{uas} \\
    \hline \hline
    Pipeline parser & ?? \\
    Joint parser    & ?? \\
    \hline
\end{tabular}
\caption{Parser evaluation on \textsc{swbd} test set. The pipeline parser, described
    in Section \ref{sec:pipeline}, uses a state-of-the-art syntax-based noisy
    channel model, followed by our parser trained on clean text. This yields
    significantly lower parse accuracies than our best joint model, which includes
additional features and the non-monotonic Edit transition.\label{tab:parseval}}
\end{table}

\begin{table}
    \centering
    \begin{tabular}{l|r}
        System                      & Disfluency $F$ \\
        \hline \hline
        Xian and Liu (2013)         & ?? \\
        Noisy channel pipeline      & ?? \\
        Joint parser                & ?? \\
        \hline
    \end{tabular}
    \caption{Disfluency evaluation on the Switchboard test set. Our joint parser-based
             model compares favourably 
             against the noisy channel model used for our pipeline experiments
             (Section \ref{sec:pipeline}) and the previous state-of-the-art system.
         \label{tab:dfl_eval}}
\end{table}

\section{Analysis of Edit behaviour}

Over Section 24, we predict 2,459 Edit transitions, which together marked 2,702
words disfluent (2,462 correctly). So the model largely assigns disfluency labels
word-by-word, only sometimes marking rightward branches disfluent.

For the 2,459 Edit transitions, in 2,309 of them S0 was disfluent (for this
statistic, there might be some of its rightward subtree that is actually fluent,
that will be marked as wrong here. We don't evaluate that here.)

Of the 2,459 Edit transitions, 1554 had no leftward children, 512 returned one word,
and another 393 together returned a total of 1080 words to the stack. Looked at
another way, 1,592 leftward children were returned to the stack, from a total of
905 Edit transitions.

Of the 1,592 instances of a word being returned to the stack, 121 were cases where
the same word was returned two or more times --- i.e. 1,471 tokens were returned
to the stack at least once, for a total of 1,592 returnings.

The 1,471 tokens break down as follows:
\begin{itemize}
    \item 607 were subsequently marked as disfluent
    \item 864 became left-children of a word marked fluent.
\end{itemize}

Of the 607 marked disfluent:
\begin{itemize}
    \item 562 were true positives
    \item 45 were false positives
\end{itemize}

Of the 864 re-integrated into the tree:
\begin{itemize}
    \item 699 were attached to their correct head
    \item 165 were attached incorrectly
\end{itemize}

Of the 165 mistakes:
\begin{itemize}
    \item 26 were false-negative disfluencies
    \item 139 were just parse errors
\end{itemize}

Some conclusions:
\begin{itemize}
\item The split for what we should do with the left children is quite even, so a
  heuristic would probably not perform well
\item 26 false negative disfluencies means that returning to the stack introduces
  few recall errors. If the word is disfluent, the model is good at making
  that separate judgment.
\item The precision of marking words that were returned to the stack as disfluent
  is similar to the overall precision.
\end{itemize}

\section{Related Work}


\section{Conclusion}


\bibliography{main}
\bibliographystyle{aclnat}

\end{document}
